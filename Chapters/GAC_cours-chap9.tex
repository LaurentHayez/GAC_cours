%---------------------------------------------------------%
%______//------             GAC             ------\\______%
%______||------         Chapitre 9          ------||______%
%______\\------   Croissance et langages (formels)   ------//______%
%---------------------------------------------------------%

\chapter{Croissance et langages (formels)}
\label{cha:croiss-et-lang-form}

  \begin{defi}
    Soit $A$ un ensemble fini, et $A^\ast$ l'ensemble de tous les mots formés à partir de $A$. $A$ s'appelle
    un \emph{alphabet}, \index{Alphabet} et $L$ est un \emph{langage} \index{Langage} sur $A$ si $L$ est un
    sous-ensemble de $A^\ast$.
  \end{defi}

  \begin{ex}
    Soit $A = \{0, 1\}$. Alors $L_1 = \{0, 1, 01, 11\}$ et $L_2 = \{0^n1^m\ |\ n, m \geq 1\}$ sont des
    langages sur $A$.
  \end{ex}


  \begin{defi}
    Un \emph{automate fini déterministe (AFD)} \index{Automate!fini déterministe} est un quintuple $(Q, A,
    \delta, q_0, F)$ constitué des éléments suivants.
    \begin{itemize}
    \item $A$ est un alphabet fini.
    \item Un ensemble fini d'états $Q$.
    \item Une fonction de transition $\delta: Q \times A \to Q$.
    \item Un état initial $q_0$.
    \item Un ensemble d'états finaux (ou acceptants) $F \subset Q$.
    \end{itemize}
  \end{defi}

  \begin{ex}
    On peut représenter un AFD par un graphe fini.
    
    \begin{center}
      \begin{tikzpicture}[shorten >=1pt, node distance=2cm, auto]
        % Nodes
        \node[state, initial]   (q0)                         {$q_0$}; 
        \node[state, accepting] (q2) [below right of=q0]     {$q_2$};
        \node[state]            (q1) [above right of=q2]     {$q_1$};
        % Path
        \path[->, >=latex] (q0) edge [bend left]  node {$a$} (q1)
                           (q1) edge [loop right] node {$b$} (q1)
                                edge [bend left]  node {$b$} (q0)
                           (q0) edge [bend right] node {$a$} (q2);
        \draw (6, -1) node{Cet automate n'est pas déterministe};
      \end{tikzpicture}
    \end{center}

    Par convention, une flèche de rien vers un cercle représente l'état initial, un sommet avec deux cercles,
    un sommet avec un seul cercle représente un état,
    représent un état final, $A = \{a, b\}$ et une flèche d'un sommet vers un autre avec une étiquette
    représente une transition et l'ensemble est $\delta: Q \times A \to Q$.

    Un mot $\omega$ est reconnu par un automate s'il existe un chemin étiqueté par $\omega$, partant de l'état
    initial et aboutissant dans un état final.

    L'automate représenté a la figure ci-dessus accepte les mots commençant par $a$, avec au moins un $b$ au
    milieu et terminant par $a$, et le mot $a$. Le langage reconnu par l'automate est l'ensemble des mots
    acceptés par l'automate.
  \end{ex}


  \begin{defi}
    Un langage est \emph{régulier} \index{Langage!régulier} s'il est accepté par un automate.
  \end{defi}

  


  \begin{prop}
    La série de croissance $f_L$ d'un langage $L$ régulier est une fonction rationelle:
      \[f_L(z) = \sum_{n \geq 0}a_nz^n\]
    où $a_n = \left| \{ \omega \in L\ |\ |\omega| = n\} \right|$. 
  \end{prop}

  \begin{preuve}[informelle, preuve par exemple]
    Considérons l'automate $\mathcal{A}$ sur $\{a, b\}$ suivant

    \begin{center}
      \begin{tikzpicture}[shorten >=1pt, node distance=3cm, auto]
        % Nodes
        \node[state, initial, accepting]   (1)                  {$1$}; 
        \node[state, accepting]            (2) [right of=1]     {$2$};
        \node[state]                       (3) [right of=2]     {$3$};
        % Path
        \path[->, >=latex] (1) edge              node {$a$} (2)
                               edge [loop above] node {$b$} (1)
                           (2) edge [bend left]  node {$b$} (1)
                               edge              node {$a$} (3)
                           (3) edge [loop above] node {$a$} (3)
                               edge [loop below] node {$b$} (3);
        
      \end{tikzpicture}
    \end{center}

    Les mots acceptés par $\mathcal{A}$ sont des mots ne contenant pas deux $a$ consécutifs. La \emph{matrice
      de voisinage} \index{Matrice de voisinage} de $\mathcal{A}$ est 
      \[
      M = 
      \begin{pmatrix}
        1 & 1 & 0 \\ 1 & 0 & 1 \\ 0 & 0 & 2
      \end{pmatrix}
      \]
    (on a une transition de $1$ vers $2$, donc $a_{12} = 1$, on a deux transitions de $3$ vers $3$, donc
    $a_{33} = 2$ etc.)

    Pour tous $q, r \in \{1, 2, 3\}$ et tous $n \in \N$, $[M^n]_{q,r}$ est le nombre de chemins de longueur
    $n$ joignant $q$ à $r$.

    \[
    M^3 = 
    \begin{pmatrix}
      3 & 2 & 3 \\ 2 & 1 & 5 \\ 0 & 0 & 8
    \end{pmatrix}
    \]
    par exemple veut dire qu'il y a deux chemins de longueur $3$ joignant $2$ à $1$ ($bbb$ ou $bab$).

    Alors $a_n = \sum_{f \in \{1, 2\}} [M^n]_{1, f}$ pour $n$ fixé. Ainsi la série est
      \[\sum_{n \geq 0}a_nz^n = \sum_{n \geq 0} \sum_{f \in \{1, 2\}}[M^n]_{1, f}z^n = \sum_{n \geq 0}
      [M^n]_{1,1}z^n + \sum_{n \geq 0} [M^n]_{1,2} z^n \]
      \[ = [I + Mz + M^2z^2 + \cdots]_{1,1} + [I + Mz + M^2z^2+\cdots]_{1,2} = [(I-Mz)^{-1}]_{1,1} +
      [(I-Mz)^{-1}]_{1,2}\]
      \[= \frac{1}{\det(I-Mz)} \left( \left[(I-Mz)^{Adj}\right]_{1,1} +
        \left[(I-Mz)^{Adj}\right]_{1,2}\right),\]
    où le déterminant est un polynômes, et les coefficients sont des polynômes. Ainsi la fonction $f_L$ est rationelle.
  \end{preuve}

  
  
  \begin{ex}
    L'automate pour les formes normales de $\F_2$ sur $\{a, A = a^{-1}, b, B = b^{-1}\}$ est représenté ci-dessous.

    \begin{center}
      \begin{tikzpicture}[shorten >=1pt, node distance=4cm, auto]
        % Nodes
        \node[state, initial, accepting]   (0)                  {$1$}; 
        \node[state, accepting]            (1) [right of=0]     {$2$};
        \node[state, accepting]            (2) [above of=0]     {$3$};
        \node[state, accepting]            (3) [left of=0]      {$2$};
        \node[state, accepting]            (4) [below of=0]     {$3$};
        % Path
        \path[->, >=latex] (0) edge              node {$a$} (1)
                               edge              node {$b$} (2)
                               edge              node {$A$} (3)
                               edge              node {$B$} (4)
                           (1) edge [loop right] node {$a$} (1)
                               edge [bend left]  node {$b$} (2)
                               edge [bend right] node {$B$} (4)
                           (2) edge [loop above] node {$b$} (2)
                               edge [bend left]  node {$A$} (3)
                               edge [bend left]  node {$a$} (1)
                           (3) edge [loop left]  node {$A$} (3)
                               edge [bend left]  node {$B$} (4)
                               edge [bend left]  node {$b$} (2)
                           (4) edge [loop below] node {$B$} (4)
                               edge [bend left]  node {$A$} (3)
                               edge [bend right] node {$a$} (1);
        
      \end{tikzpicture}
    \end{center}
    
    Commme on représenter le langage par un automate fini déterministe, le langage est régulier. Ainsi la
    série de croissance de $\F_2$ par rapport à $\{a, A, b, B\}$ est rationelle.

    On a calculé que $a_n = 4 \cdot 3^{n-1}$, et
      \[f_{(G, S)}(z) = 1 + \sum_{n \geq 1} 4\cdot 3^{n-1}z^n = 1 \frac{4}{3} \sum_{n \geq 1}(3z)^n = 1 +
      \frac{4}{3} \frac{1}{1-3z}.\]
  \end{ex}

  

  







%%% Local Variables:
%%% mode: latex
%%% TeX-master: "../GAC_cours.tex" 
%%% End: