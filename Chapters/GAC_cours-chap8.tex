%---------------------------------------------------------%
%______//------             GAC             ------\\______%
%______||------         Chapitre 8          ------||______%
%______\\------   Propriétés géométriques   ------//______%
%---------------------------------------------------------%

\chapter{Propriétés géométriques}
\label{cha:propr-geom}

  Une propriété géométrique est une propriété invariante par quasi-isométries.


  \begin{defi}\index{Groupes!commensurables} 
    Deux groupes sont \emph{commensurables} s'ils possèdent des sous-groupes d'indice fini isomorphes.
  \end{defi}

  \begin{rems}
    \begin{enumerate}
    \item Pour les groupes finiment engendrés, deux groupes commensurables sont quasi-isométriques (à cause du
      corollaire précédent).
    \item En revanche, deux groupes quasi-isométriques n'implique pas qu'ils sont commensurables (en général).
    \end{enumerate}
  \end{rems}

  \begin{exs}
    \begin{enumerate}
    \item Si $F$ est un groupe fini, alors $\Z \times F$ et $D_\infty$ sont commensurables, car ils possèdent
      tous les deux le sous-groupe $\Z$ qui est d'indice fini.

    \item Pour $k, l \geq 2$, $\F_k$ est commensurable à $\F_l$.
    \end{enumerate}
  \end{exs}

  \begin{defi}
    Soit $(P)$ une propriété des groupes finiment engendrés. On dit que $G$ est \emph{virtuellement $(P)$}
    \index{Groupe!virtuellement $(P)$} si $G$ possède un sous-groupe $H$ d'indice fini qui a la propriété $(P)$.
  \end{defi}

  \begin{ex}[virtuellement libre]
    Si $G$ est virtuellement libre, alors $G$ possède un sous-groupe $H$ d'indice fini qui est libre.
  \end{ex}

  \begin{exs}
    \begin{enumerate}
    \item \og Être fini \og est une propriété géométrique. Car $G$ est fini ss'il est qi à $\{1\}$ et par
      transitivité, si $H$ est qi à $G$, il est aussi qi à $\{1\}$ et donc fini.

    \item \og Être cyclique infini\fg{}, c'est-à-dire \og être $\Z$\fg{}, n'est pas une propriété
      géométrique. $\Z$ et $\Z \times \Z/2\Z$ sont qi, mais le second n'est pas cyclique infini. De même pour
      $\Z$ et $D_\infty$.
    \end{enumerate}
  \end{exs}


  \begin{prop}
    \begin{enumerate}
    \item Être virtuellement $\Z$ est une propriété géométrique.
    \item Être virtuellement libre est une propriété géométrique.
    \item Être virtuellement abélien est une propriété géométrique.
    \item Avoir une présentation finie est une propriété géométrique.
    \item Avoir un problème des mots résolubles est une propriété géométrique.
    \item Être virtuellement nilpotent est une propriété géométrique.
    \end{enumerate}
  \end{prop}

  La proposition est difficile à prouver, mais pour quelques assertions, on peut le montrer en utilisant la
  croissance des groupes.

  \section{Croissance des groupes}
  \label{sec:croiss-des-groupes}
  
    Soit $G$ un groupe finiment engendré et $S = S^{-1}$ une partie finie symétrique génératrice de $G$.

    \begin{defi}
      La \emph{fonction de croissance} \index{Fonction!de croissance} de $G$ par rapport à $S$ est
        \[V_S: \N \to \R^+\ (\N^+),\ n \mapsto |B_S(n)| \]
      où $B_S(n) = \{g \in G\ |\ |g|_S \leq n\}$.
    \end{defi}

    \begin{exs}
      \begin{enumerate}
      \item Si $G$ est fini, alors $V_S(n)$ est constant pour $n \gg 0$

      \item Si $G = \Z$ et $S = \{\pm 1\}$,
        \begin{center}
          Dessin ici
        \end{center}
        alors $V_s(n) = 2n + 1$.

      \item Si $G = \Z^2$ avec $S = \{(\pm 1, 0), (0, \pm 1)\}$,
        \begin{center}
          Dessin ici
        \end{center}
        alors $V_S(n) = 1 + 4 \sum_{j=1}^n(n+1-j) = 2n^2 + 2n + 1 \leq (2n+1)^2$ (à vérifier).

      \item Si $G = \Z^d$ avec $S = \{(\pm 1, 0, \ldots, 0), \ldots, (0, 0, \ldots, \pm 1)\}$. Alors
        \[|B_S(n)| \leq (2n+1)^d,\ |B_S(n)| \geq \text{volume de la boule euclidienne de rayon }
        \frac{n}{\sqrt{d}} \cong C_dn^d.\]

      \item Si $G = \F_k$, $S = \{a_1^{\pm 1}, \ldots, a_k^{\pm 1}\}$. Soit $S(n) = \{g \in \F_k\ |\ |g|_S =
        n\}$. Alors $|S(n)| = 2k(2k-1)^{n-1}$ (car on a $2k$ choix pour la première lettre du mot, et ensuite
        comme on ne considère que des mots réduits, on a $2k-1$ choix pour le reste des lettres du mot). Ainsi
          \[V_S(n) = \sum_{i=0}^n |S(i)| = \cdots = \frac{k(2k-1)^n-1}{k-1}.\]
       \item Le groupe de Heisenberg discret.\index{Groupe!de Heisenberg discret}
         Si $A$ est un anneau commutatif à unité, le groupe de Heisenberg sur $A$ est 
         \[Heis(A) = \left\{
           \begin{pmatrix}
             1 & x & z \\ 0 & 1 & y \\ 0 & 0 & 1
           \end{pmatrix}\ |\ x,y,z \in A \right\} \subseteq GL_3(A).
         \]
         Considérons $Heis(\Z) = \langle a,b,c \rangle$.
           \[a =
           \begin{pmatrix}
             1 & 1 & 0 \\ 0 & 1 & 0 \\ 0 & 0 & 1
           \end{pmatrix},\ 
           b =
           \begin{pmatrix}
             1 & 0 & 0 \\ 0 & 1 & 1 \\ 0 & 0 & 1
           \end{pmatrix},\ 
           c =
           \begin{pmatrix}
             1 & 0 & 1 \\ 0 & 1 & 0 \\ 0 & 0 & 1
           \end{pmatrix}.
           \]
         On a que $[a,b] = c$, $[a,c] = [b, c] = 1$ (commutateurs). Ceci nous dit que $c$ commute avec $a$ et $b$.
         
         \textbf{Exercice:} Pour tous $m, n \in \Z$, $a^mb^n = c^{mn}b^na^n$ $(\ast)$ . Indication: démontrer que $a^m b
         = c^m ba^m$.
      \end{enumerate}
    \end{exs}

    \begin{rem}
      $(\ast)$ nous donne que tout $g \in G$ a une expression $g = a^mb^nc^p$ pour $m, n, p \in \Z$.
    \end{rem}

    \begin{prop}[Croissance du groupe de Heisenberg]
      Pour le groupe de Heisenberg, il existe des polynômes $P_1, P_2$ de degré 4 tels que
        \[\forall n \in \N, P_1(n) \leq V_S(n) \leq P_2(n).\]
    \end{prop}

    \begin{preuve}
      \begin{enumerate}
      \item Montrons que si si $g = a^m b^n c^p$, et $|g|_S \leq N$, alors $|m|, |n| \leq N$, $|p| \leq
        N^2$. En effet on au plus $(2N+1)$ choix pour $|m|$ et $|n|$ et $2N^2 + 1$ choix pour $|p|$, donc
          \[V_S(n) \leq (2N+1)^2(2N^2+1).\]
      \end{enumerate}
    \end{preuve}

    




%%% Local Variables:
%%% mode: latex
%%% TeX-master: "../GAC_cours.tex" 
%%% End: