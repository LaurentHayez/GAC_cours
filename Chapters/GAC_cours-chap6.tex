%---------------------------------------------------------%
%______//------             GAC             ------\\______%
%______||------         Chapitre 6          ------||______%
%______\\------  Transformations de Tietze  ------//______%
%---------------------------------------------------------%

\chapter{Transformations de Tietze}
  
  \begin{defi}
  Soit $\langle X_1 , \ldots, X_n | \underbrace{r_1, \ldots, r_m}_{R}\rangle$ une présentation finie d'un
  groupe $G$. Les transformations suivantes, appelées \emph{transformation de Tietze} \index{Transformation de
    Tietze}, changent la présentation sans changer le groupe.
  \end{defi}



   %Algorithme 
   \begin{algorithm}
     \caption{Première transformation de Tietze}
     \label{alg:trans-tietze-1}
     \begin{algorithmic}
       \State $T_1$ ou $R^+$: Ajouter à la présentation de $G$ un relateur $r_{m+1}$ qui appartient à la
       clotûre normale de $R$ (notée $\overline{R}$, $\lhd R \rhd$ ou $gp_G(R)$).
       \State Soit $r_{m+1} \in \overline{R} \setminus R$: $\langle X | R \rangle \xrightarrow{R^+, T_1}
       \langle X | R \cup \{r_{m+1}\} \rangle$.
     \end{algorithmic}
   \end{algorithm}


   \begin{ex}
     Considérons $\Z^2 = \langle a, b | aba^{-1}b^{-1} \rangle \xrightarrow{R^+} \langle a, b | [a,b], [a,b]^2
     \rangle$.
   \end{ex}


   \begin{algorithm}
     \caption{Deuxième transformation de Tietze}
     \label{alg:trans-tietze-2}
     \begin{algorithmic}
       \State $R^-$: Opération inverse de $R^+$.
       \State Soit $r \in R \setminus \overline{R \setminus \{r\}}$. Alors $\langle X | R \rangle
       \xrightarrow{R^-} \langle X | R \setminus \{r\} \rangle$.
     \end{algorithmic}
   \end{algorithm}


   \begin{algorithm}
     \caption{Troisième transformation de Tietze}
     \label{alg:trans-tietze-3}
     \begin{algorithmic}
       \State $X^+$: Ajouter à la présentation de $G$ un générateur $x_{n+1}$ ainsi qu'une relation $x_{n+1} = w(x_1,
       \ldots, x_n)$ (un mot sur $x_1, \ldots, x_n$).
       \State $\langle X | R \rangle \xrightarrow{X^+} \langle X, x_{n+1} | R \cup \{ x_{n+1}w^{-1}(x_1, \ldots, x_n)\} \rangle$
     \end{algorithmic}
   \end{algorithm}


   \begin{algorithm}
     \caption{Quatrième transformation de Tietze}
     \label{alg:trans-tietze-4}
     \begin{algorithmic}
       \State $X^-$: Opération inverse de $X^+$.
       \State Soit $y \in X$, $w \in \langle X \setminus \{y\} \rangle$ et $y^{-1}w$ est le seul mot dans $R$
       qui contient $y$. Alors
       \State $\langle X | R \rangle \xrightarrow{X^-} \langle X \setminus \{y\} | R \setminus \{y^{-1}w\} \rangle$.
     \end{algorithmic}
   \end{algorithm}

   \begin{ex}
     Soit $G = \langle x,y | xyx = yxy \rangle$. C'est le groupe fondamental du noeud de trèfle. On va
     utiliser les transformations de Tietze. On a
     \begin{align*}
       \langle x,y | xyx = yxy \rangle &\xrightarrow{X^+} \langle x,y,a,b | xyx = yxy, a=xy, b=xyx \rangle\\
       &\xrightarrow{R^+} \langle x,y,a,b | xyx=yxy, a=xy, b=xyx, x=a^{-1}b, y = b^{-1} b^{-1}a^2, a^3 = b^2
         \rangle & a^3 = xyxyxy\\
       &\xrightarrow{R^-} \langle x,y,a,b | a^3 = b^2, x = a^{-1}b, y = b^{-1}a^2 \rangle\\
       &\xrightarrow{X^-} \langle a,b | a^3 = b^2 \rangle.
     \end{align*}
     Cette dernière présentation correspond au produit libre amalgamé.
   \end{ex}


   \begin{prop}[de \textsc{Tietze}] \label{prop:de-Tietze} \index{Proposition!de \textsc{Tietze}}
     Les transformations de Tietze ne changent pas le groupe.
   \end{prop}

   \begin{preuve}[pour $X^+$]
     Supposons que $G = \langle X | R \rangle$, $y$ est un symbole qui n'est pas dans $X$, et $w(X)$ un mot
     réduit de $\F(X)$. On veut montrer que $\langle X, y | R \cup \{y^{-1}w(X)\} \rangle \cong \langle X, R
     \rangle = G$. 

     Soit $\phi : \F(X) \to G$ l'homomorphisme donné par la propriété universelle des groupes libres. Le
     groupe libre $\F(X, y)$ sur $X \cup \{y\}$ est engendré librement par $X \cup
     \{y^{-1}w(X)\}$. C'est-à-dire que $\F(X \cup \{y\}) = \F(X \cup \{y^{-1}w(X)\})$. C'est vrai car à
     partir de $y^{-1}w(X)$, on peut obtenir $y$ (cette inclusion est sensée être facile), et à partir de
     $y$ on peut obtenir $y^{-1}w(X)$. Ainsi on a
       \[X \cup \{y^{-1}w(x)\} \hookrightarrow \F(x,y) = \F(X \cup \{y^{-1}w(X)\}.\]
     Il y a un unique homomorphisme $\phi^1: \F(x,y) \to G$ tel que $\phi^1(x) = \phi(x)$ et
     $\phi^1(y^{-1}w(X)) = 1$ pour $x \in X$.

     L'homomorphisme $\phi^1: \F(X,y) \to G$ se factorise comme $F(X,y) \xrightarrow{\chi} \F(X)
     \xrightarrow{\phi} G$ où $\chi(x) = x$ pour tout $x \in X$ et $\chi(y) = w(x)$. Alors $\phi^1$ est
     surjective et
       \[\ker \phi^1 = \chi^{-1}(\phi^{-1}(1)) = \chi^{-1} \left( gp_{\F(X)}R \right) = gp_{\F(X,y)}(R \cup
       \{y^{-1}w(X)\}).\]
     Ainsi par le premier théorème d'isomorphisme, on a que 
       \[G \cong \F(x,y)/\ker \phi^1 = \langle X,y | R \cup \{y^{-1}w(X)\} \rangle.\]
   \end{preuve}



   \begin{theo}[de \textsc{Tietze}] \label{thm:de-Tietze} \index{Théorème!de \textsc{Tietze}}
     Soient ${\cal P}_1 = \langle X | R \rangle$ et $ {\cal P}_2 = \langle Y | S \rangle$ des présentations
     finies pour un groupe $G$. Alors il existe une suite finie de transformations de Tietze qui transforment
     ${\cal P}_1$ en ${\cal P}_2$.
   \end{theo}

   \begin{preuve}
     cf. feuille annexe. $G$ est donné par ${\cal P}_1$ et ${\cal P}_2$. Alors chaque $x \in X$ peut être
     écrit comme un mot sur $Y$, et on note $x(Y)$. Alors $X(Y)$ représente tous les mots sur $Y$ qui
     décrivent les éléments de $X$. De la même manière, on définit $y(X)$ et $Y(X)$. 

     On commence avec ${\cal P}_1$ et on utilise les transformations suivantes (voir feuille annexe).

     Intuitivement, on ajoute tous les générateurs $Y$ et on enlève tous les générateurs $X$. On utilise les
     transformations $R^+$ $|X|+|R|+|Y|+|S|$ fois et $R^- 2(|R|+|Y|)$ fois. Donc il y a un nombre fini de transformations de ${\cal
       P}_1$ à ${\cal P}_2$.
   \end{preuve}

   \begin{cor}
     On peut énumérer toutes les présentations finies d'un groupe $G$ à partir d'une présentation quelconque
     pour $G$.
   \end{cor}

   \begin{prop}
     Si le groupe $G$ a une présentation finie $\langle X_1 |R \rangle$ et une présentation infinie $\langle
     X_2 | S \rangle$ où $S$ est infini, alors il existe un entier $n$ tel que $\langle X_2 | s_1, \ldots, s_n
     \rangle$ est une présentation finie pour $G$.
   \end{prop}




%%% Local Variables:
%%% mode: latex
%%% TeX-master: "../GAC_cours.tex" 
%%% End: