%---------------------------------------------------------%
%______//------             GAC             ------\\______%
%______||------         Chapitre 1          ------||______%
%______\\------       Groupes Libres        ------//______%
%---------------------------------------------------------%

\chapter{Groupes libres}

  Soit $A$ un alphabet, fini ou infini.
  \begin{itemize}
  \item On considère l'ensemble des mots de longueur finie sur $A \cup A^{-1}$ (on introduit pour chaque
    nouvelle lettre $a \in A$ une nouvelle lettre $a^{-1}$).
  \item Un mot est \emph{réduit} s'il ne contient aucune expression de la forme $aa^{-1}$ ou $a^{-1}a$, $a \in
    A$.
  \item Le \emph{mot vide} est réduit et se note $1$ (ou $\epsilon$ ou $e$,....).
  \end{itemize}
  
  \begin{defi} \label{defi-1-grp-libre}
    Le \emph{groupe libre sur $A$}, noté $\Fa$ est l'ensemble des mots réduits sur $A \cup A^{-1}$. Ceci
    définit $\Fa$ comme ensemble. Pour avoir un groupe il faut définir le produit: c'est la
    concaténation/réduction. On écrit deux mots réduits bouts à bouts, puis on réduit en supprimant les
    apparitions de $aa^{-1}$ ou $a^{-1}a$. Avec ce produit, $\Fa$ est un groupe.

    Si $A = \{a_1, \ldots, a_n\}$, on note $\F_n = \Fa$ et on parle du \emph{groupe libre de rang $n$}.
  \end{defi}

  \begin{exercice}
    Montrer que $\F_1 = \Z$. En fait, on a $A = \{a\}$, donc les mots sont $aaa\cdots a^{-1}$, c'est-à-dire
    $a^n$ ou $a^{-n}$.
  \end{exercice}

  \begin{rem}
    $\F_1 = \Z$ et $\F_n$ ($n > 1$) ont des propriétés très différentes.
  \end{rem}

  \begin{defi} \label{defi-2-grp-libre}
    Soit $X$ un alphabet fini. Le \emph{monoïde libre} sur $X$, noté $M(X)$, est l'ensemble des mots sur $X$
    avec le produit donné par la concaténation. Soit $X = A \cup A^{-1}$. Nous pouvons poser sur $M(X)$ la
    relation d'équivalence suivante: $w_1 \sim w_2 \iff $ après réduction, $w_1 = w_2$. Le quotient
    $M(X)/\sim$ est le \emph{groupe libre} $\Fa$, où l'inverse de la classe d'équivalence de $x_1^{\epsilon_1}
    \cdots x_n^{\epsilon_n}$ est la classe d'équivalence de $x_n^{-\epsilon_n} \cdots x_1^{-\epsilon_1}$ avec
    $\epsilon_i \in \Z$ pour tout $i$. L'opération est la concaténation (la réduction est implicite).
  \end{defi}

  On fait souvent l'abus de language suivant: on va identifier un mot réduit avec sa classe d'équivalence.

  \begin{prop}
    \begin{enumerate}
    \item $\Fa$ est un groupe (von Dyck, 1882).
    \item La définition \ref{defi-1-grp-libre} est équivalente à la définition \ref{defi-2-grp-libre}.
    \end{enumerate}
  \end{prop}

  \begin{preuve}
    \begin{enumerate}
    \item
      \begin{itemize}
      \item Le neutre est le mot vide, noté $\epsilon$ ou $1_{\Fa}$.
      \item L'inverse de $a_1^{\epsilon_1} \cdots a_n^{\epsilon_n}$ est $a_n^{-\epsilon_n} \cdots
        a_1^{-\epsilon_1}$.
      \item L'opération de concaténation et réduction est associative (exercice)
      \end{itemize}

    \item Exercice. \qedhere
    \end{enumerate}
  \end{preuve}

  \paragraph{Question:} pourquoi dit-on  que $\Fa$ est libre sur $A$?
  \paragraph{Réponse:} car tout mot réduit sur $A$ représentant l'élément neutre est le mot vide
  (exercice). Alors il n'y a pas de relation entre les lettres dans $A$, et $\Fa$ à la présentation $\langle
  a_1, a_2, \ldots, a_n | - \rangle$.

  \section{Propriété universelle du groupe libre (PU)}
  \label{sec:propriete-universelle}
  
    Soit $G$ un groupe et $f:A \to G$ une application. Alors il existe un unique homomorphisme $\phi$ tel que
    le diagramme suivant commute.
    \begin{center}
      \begin{tikzpicture}
        \node (A) at (0,0) {$A$};
        \node (FA) at (3, 0) {$\Fa$};
        \node (G) at (1.5, -1.5) {$G$};
        \draw[right hook-latex] (A) to node[midway, above]{$i$} (FA);
        \draw[->, >=latex] (A) to node[midway, below left]{$f$} (G);
        \draw[->, >=latex] (FA) to node[midway, below right]{$!\phi$} (G);
      \end{tikzpicture}
    \end{center}
    Ceci signifie que toute application $f:A \to G$ s'étend en un unique homomorphisme $\phi: \Fa \to G$ où
    pour $w = a_{i_1}^{\epsilon_1} \cdots a_{i_n}^{\epsilon_n}$ on pose $\phi(w) = f(a_{i_1})^{\epsilon_1}
    \cdots f(a_{i_n})^{\epsilon_n}$ avec $\epsilon_i \in \Z$. En particulier, si $A$ est une partie
    génératrice de $G$ (par exemple $A = G$), on voit que $\Fa$ se surjecte sur $G$ et ceci nous donne le
    théorème suivant, qui est très important.

    \begin{theo} \label{theo-tt-grp-quotient-grp-libre}
      Tout groupe est quotient d'un groupe libre.
    \end{theo}

    \begin{preuve}
      Si $A$ est une partie génératrice d'un groupe $G$, par le premier théorème d'isomorphisme, il existe un
      isomorphisme tel que $\phi: \Fa \to G$ implique que $\Fa/\ker \phi \cong \im \phi = G$.
    \end{preuve}




  




%%% Local Variables:
%%% mode: latex
%%% TeX-master: "../GAC_cours.tex" 
%%% End: