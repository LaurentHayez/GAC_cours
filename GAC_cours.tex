\documentclass[a4paper, 12pt, usenames, dvipsnames, chapterprefix=true]{scrreprt}

\usepackage[utf8]{inputenc}
\usepackage[T1]{fontenc}
\usepackage{graphicx, wrapfig}
\usepackage{lmodern}
\usepackage{fancyhdr}
\usepackage{color, colortbl}
\usepackage{xcolor}
\usepackage{amsmath, amssymb, mathrsfs, amsthm, thmtools, MnSymbol}
\usepackage[framemethod=tikz]{mdframed}
\usepackage{pgf, pgfplots, tikz, pst-solides3d}
\usetikzlibrary{cd} %To draw commutative diagrams
\usetikzlibrary{arrows}
\usetikzlibrary{shapes}
\usepackage[chapter]{algorithm}
\usepackage{algorithmicx, algpseudocode}
\usepackage{listings}
\usepackage{multicol, multirow}
% the following patch corrects a bug for the closing parenthesis  
\usepackage{etoolbox}
\makeatletter
\patchcmd{\lsthk@SelectCharTable}{%
  \lst@ifbreaklines\lst@Def{`)}{\lst@breakProcessOther)}\fi}{}{}{}
\makeatother
\usepackage{hyperref}
\usepackage{todonotes}
\usepackage{makeidx}
\usepackage[inline]{enumitem}
\usepackage[francais]{babel}
\usepackage{caption, tabu}


\newcommand{\N}{\mathbb{N}}
\newcommand{\Q}{\mathbb{Q}} 
\newcommand{\R}{\mathbb{R}}
\newcommand{\Z}{\mathbb{Z}}
\newcommand{\F}{\mathbb{F}}
\newcommand{\Fa}{\F(A)} 
\newcommand{\C}{\mathbb{C}}
\newcommand{\K}{\mathbb{K}}
\renewcommand{\epsilon}{\varepsilon}
\renewcommand{\phi}{\varphi}
\renewcommand{\emph}{\textbf}
\newcommand{\im}{\mathrm{Im}}
\newcommand{\rev}[2]{\substack{#1\\\downarrow\\ #2}p}

\DeclareMathOperator{\rk}{rk}

\synctex=1



%%%%%%%%	Définitions des environnements de théorèmes	%%%%%%%%
%----- ENVIRONNEMENT POUR LES DÉFINITIONS ----%
\declaretheoremstyle[
  spaceabove=0pt, spacebelow=0pt, headfont=\normalfont\bfseries\scshape,
    notefont=\mdseries, notebraces={(}{)}, headpunct={. }, headindent={},
    postheadspace={ }, postheadspace=4pt, bodyfont=\normalfont, %qed=$$,
    mdframed={
      leftmargin=-5,
      rightmargin=-5,
      middlelinewidth=1pt,
      roundcorner=5pt,
      middlelinecolor=OliveGreen,
      innerlinecolor=OliveGreen,
      outerlinecolor=OliveGreen,
      % apptotikzsetting={\tikzset{mdfbackground/.append style ={
      %       shade, left color=OliveGreen!20, right color = OliveGreen!20}}}
   }
]{defstyle}

\declaretheorem[style=defstyle, numberwithin=chapter, title=Définition]{defi}
%________________________________________________________


%----- ENVIRONNEMENT POUR LES THEOREMES ----%
\declaretheoremstyle[
  spaceabove=0pt, spacebelow=0pt, headfont=\normalfont\bfseries\scshape,
    notefont=\mdseries, notebraces={(}{)}, headpunct={. }, headindent={},
    postheadspace={ }, postheadspace=4pt, bodyfont=\normalfont\itshape, %qed=$$,
    mdframed={
      leftmargin=-5,
      rightmargin=-5,
      middlelinewidth=1pt,
      roundcorner=5pt,
      middlelinecolor=OliveGreen,
      innerlinecolor=OliveGreen,
      outerlinecolor=OliveGreen,
      % apptotikzsetting={\tikzset{mdfbackground/.append style ={
      %       shade, left color=OliveGreen!20, right color = OliveGreen!20}}}
   }
]{thmstyle}
\declaretheorem[style=thmstyle, sibling=defi, title=Théorème]{theo}
\declaretheorem[style=thmstyle, sibling=defi, title=Corollaire]{cor}
\declaretheorem[style=thmstyle, sibling=defi, title=Proposition]{prop}
\declaretheorem[style=thmstyle, sibling=defi, title=Propriétés]{propri}
\declaretheorem[style=thmstyle, sibling=defi, title=Observation]{obs}
\declaretheorem[style=thmstyle, sibling=defi, title=Observations]{obss}
\declaretheorem[style=thmstyle, sibling=defi, title=Lemme]{lem}
\declaretheorem[style=thmstyle, sibling=defi, title=Conséquence]{conseq}
%_________________________________________________________

%----- ENVIRONNEMENT POUR LES PREUVES ----%
\declaretheoremstyle[
  spaceabove=0pt, spacebelow=0pt, headfont=\normalfont\bfseries\scshape,
    notefont=\mdseries, notebraces={(}{)}, headpunct={. }, headindent={},
    postheadspace={ }, postheadspace=4pt, bodyfont=\normalfont, 
    mdframed={
      leftmargin=15,
      rightmargin=15,
      hidealllines=true,
      font=\small
   }
]{preuvestyle}

\declaretheorem[style=preuvestyle, numbered = no, title=Preuve, qed=\textcolor{OliveGreen!80}{\qedsymbol}]{preuve}
\declaretheorem[style=preuvestyle, title=Exercice, numberwithin=chapter, qed=\textcolor{OliveGreen!80}{$\spadesuit$}]{exercice}
\declaretheorem[style=preuvestyle, sibling=defi, title=Remarque, qed =
\textcolor{OliveGreen!80}{$\clubsuit$}]{rem}
\declaretheorem[style=preuvestyle, sibling=defi, title=Remarques, qed = \textcolor{OliveGreen!80}{$\clubsuit$}]{rems}
%________________________________________________________
%----- ENVIRONNEMENT POUR LES EXEMPLES ----%
\declaretheoremstyle[
  spaceabove=0pt, spacebelow=0pt, headfont=\normalfont\bfseries\scshape,
    notefont=\mdseries, notebraces={(}{)}, headpunct={. }, headindent={},
    postheadspace={ }, postheadspace=4pt, bodyfont=\normalfont, qed=\textcolor{OliveGreen!80}{$\bigstar$},
    mdframed={
      leftmargin=15,
      rightmargin=15,
      font=\small,
      outerlinewidth=1pt,
      innerlinewidth=1pt,
      middlelinewidth=1pt,
      hidealllines=true, leftline=true,
      innerlinecolor=OliveGreen!80,
      outerlinecolor=OliveGreen!80,
      middlelinecolor=White,
   }
]{exstyle}

\declaretheorem[style=exstyle, numberlike=defi, title=Exemple]{ex}
\declaretheorem[style=exstyle, numberlike=defi, title=Exemples]{exs}
%________________________________________________________



\addtokomafont{disposition}{\normalfont\bfseries}

\title{\normalfont{\bfseries{Groupes, algorithmique et combinatoire: cours 2015}}}
\author{Laurent \textsc{Hayez}}
\date{Date de création: 30 septembre 2015\\ Dernière modification: \today}

\makeindex

\begin{document}


\renewcommand{\labelitemi}{\textbullet}

\tikzset{math3d/.style=
{x= {(-0.353cm,-0.353cm)}, y={(1cm,0cm)}, z={(0cm,1cm)}}}


\maketitle


%Table of contents
\tableofcontents

\setcounter{chapter}{-1}

\part{Objets}

% Chapter 0: Motivations
%---------------------------------------------------------%
%______//------             GAC             ------\\______%
%______||------         Chapitre 0          ------||______%
%______\\------         Motivations         ------//______%
%---------------------------------------------------------%

\chapter{Motivations}

  %Déf 0.1
  \begin{defi}
    Soit $G$ un groupe muni d'une loi ``$\cdot$''. $G$ est un \emph{groupe} si 
    \begin{enumerate}
    \item il existe un élément neutre $e \in G$;
    \item pour chaque élément $g \in G$, il existe un inverse $g^{-1}$;
    \item $\cdot$ est associative: $(x\cdot y)\cdot z = x \cdot (y \cdot z)$.
    \end{enumerate}
  \end{defi}

  \begin{exs}
    \begin{enumerate}
    \item $G = \{e\}$.
    \item $G = (\Z, +)$.
    \item $G = \Z/n\Z$.
    \item $G = S_3$ le groupe des symétries d'un triangle.
    \item $G = D_4$ le groupe des symétries d'un carré.
    \end{enumerate}
  \end{exs}

  \section{Algorithmes et combinatoire?}

    Chaque groupe $G$ admet un présentation 
      \[G = \langle X | R \rangle \]
    où $X \subset G$ est une partie génératrice et $R$ est un ensemble de relations.

    \begin{exs}
      \begin{enumerate}
      \item $\Z = \langle a = 1 | - \rangle$.
      \item $S_3 = \langle t_1, t_2 | t_1^2 = e = t_2^2, (t_1t_2)^3 = e\rangle$.
      \item $D_4 = \langle x, y | x^2 = y^4 = (xy)^2 = e \rangle$.
      \item $\Z/7\Z = \langle x | x^7 = e\rangle$.
      \end{enumerate}
    \end{exs}
    
    Attention: la présentation n'est pas unique, car par exemple $\Z = \langle a, b | b = 1 \rangle$.

  \section{Problèmes de Dehn}

    \subsection{Problème de l'égalite (PE)}

      Existe-t-il un algorithme permettant de décider pour tout couple de mots $(u,v)$ sur $X$ (pour un groupe
      $G = \langle X | R \rangle$) s'ils représentent le même élément du groupe ($u =_G v$)?

      Par exemple, soit $G = \langle x,y,z |  x^2yx^{-1}z = x^3y^3\rangle$. Est-ce que $xyx^{-1}z =_G
      zx^2y^{-1}z$? Ou par exemple dans $S_3$, est-ce que $t_1t_2t_1^3t_2 =_{S_3} t_2t_1$? En fait, oui car
      \begin{align*}
        t_1t_2t_1^3t_2 &= t_1t_2t_1t_1^2t_2 & t_1^2 = e\\
        &= t_1t_2t_1t_2 & (t_1t_2)^3 = e\\
        &= t_2^{-1}t_1{-1} & t_1 = t_1^{-1},\ t_2 = t_2^{-1}\\
        &= t_2t_1.
      \end{align*}

    \subsection{Problème des mots (PM)}

      Existe-t-il un algorithme permettant de décider pour tout mot $w$ sur $X$ si $w =_G e$?

      Si $G = \langle X | R \rangle$, on peut dessiner son graphe de Cayley, qui est un espace métrique. Les
      sommets de ce graphe sont $\{g \in G\}$ et les arêtes sont $\{(g, gx)\, :\, g \in G,\, x \in X\}$.

      \begin{exs}
        \begin{enumerate}
        \item Considérons par exemple 
            \[S_3 = \{e, (12) = t_1, (23) = t_2, (13) = t_3, (123)=t_4, (132) = t_5\}.\]
          On a $X = \{t_1, t_2\}$. Le graphe de Cayley est
          \begin{center}
            \begin{tikzpicture}
              \draw (0,0) node[scale=0.8]{$\bullet$} node[above right]{$t_4$};
              \draw (4,-3) node[scale=0.8]{$\bullet$} node[right]{$t_3 = t_5t_2$};
              \draw (3,-2) node[scale=0.8]{$\bullet$} node[below left]{$t_5$};
              \draw (-4,-3) node[scale=0.8]{$\bullet$} node[below left]{$t_2$};
              \draw (-3,-2) node[scale=0.8]{$\bullet$} node[above left]{$e$};
              \draw (0,1) node[scale=0.8]{$\bullet$} node[right]{$t_1 = t_4t_2$};
              \draw[color = ForestGreen] (0,1) -- (0,0);
              \draw[color = ForestGreen] (4,-3) -- (3,-2);
              \draw[color = ForestGreen] (-4,-3) -- (-3,-2);
              \draw[color = red] (0,0) to[bend left] (4,-3);
              \draw[color = red] (3,-2) to[bend left] (-4,-3);
              \draw[color = red] (-3,-2) to[bend left] (0,1);
            \end{tikzpicture}
          \end{center}

        \item Considérons $\Z = \langle 2, 3 | 2+2+2 = 3+3\rangle$. Dessinons son graphe de Cayley.
          \begin{center}
    
            \begin{tikzpicture}
              \foreach \k in {-3,-2,...,5} { \draw (\k, 0) node[scale=0.8]{$\bullet$} node[below]{$\k$}; }
              \foreach \k in {-3, -2, ..., 3}{ \draw[color = red] (\k, 0) to[bend left] ({\k+2}, 0); }
              \foreach \k in {-3, -2, ..., 4}{ \draw[color = ForestGreen] (\k, 0) to[bend right] ({\k+3}, 0); }
            \end{tikzpicture}

          \end{center}
          En fait on dit que ce groupe est quasi-isométrique à $\Z = \langle 1 | - \rangle$.
        \end{enumerate}
      \end{exs}



  




%%% Local Variables:
%%% mode: latex
%%% TeX-master: "../GAC_cours.tex" 
%%% End:

% Chapter 1: Groupes Libres
%---------------------------------------------------------%
%______//------             GAC             ------\\______%
%______||------         Chapitre 1          ------||______%
%______\\------       Groupes Libres        ------//______%
%---------------------------------------------------------%

\chapter{Groupes libres}

  Soit $A$ un alphabet, fini ou infini.
  \begin{itemize}
  \item On considère l'ensemble des mots de longueur finie sur $A \cup A^{-1}$ (on introduit pour chaque
    nouvelle lettre $a \in A$ une nouvelle lettre $a^{-1}$).
  \item Un mot est \emph{réduit} s'il ne contient aucune expression de la forme $aa^{-1}$ ou $a^{-1}a$, $a \in
    A$.
  \item Le \emph{mot vide} est réduit et se note $1$ (ou $\epsilon$ ou $e$,....).
  \end{itemize}
  
  \begin{defi} \label{defi-1-grp-libre}
    Le \emph{groupe libre sur $A$}, noté $\Fa$ est l'ensemble des mots réduits sur $A \cup A^{-1}$. Ceci
    définit $\Fa$ comme ensemble. Pour avoir un groupe il faut définir le produit: c'est la
    concaténation/réduction. On écrit deux mots réduits bouts à bouts, puis on réduit en supprimant les
    apparitions de $aa^{-1}$ ou $a^{-1}a$. Avec ce produit, $\Fa$ est un groupe.

    Si $A = \{a_1, \ldots, a_n\}$, on note $\F_n = \Fa$ et on parle du \emph{groupe libre de rang $n$}.
  \end{defi}

  \begin{exercice}
    Montrer que $\F_1 = \Z$. En fait, on a $A = \{a\}$, donc les mots sont $aaa\cdots a^{-1}$, c'est-à-dire
    $a^n$ ou $a^{-n}$.
  \end{exercice}

  \begin{rem}
    $\F_1 = \Z$ et $\F_n$ ($n > 1$) ont des propriétés très différentes.
  \end{rem}

  \begin{defi} \label{defi-2-grp-libre}
    Soit $X$ un alphabet fini. Le \emph{monoïde libre} sur $X$, noté $M(X)$, est l'ensemble des mots sur $X$
    avec le produit donné par la concaténation. Soit $X = A \cup A^{-1}$. Nous pouvons poser sur $M(X)$ la
    relation d'équivalence suivante: $w_1 \sim w_2 \iff $ après réduction, $w_1 = w_2$. Le quotient
    $M(X)/\sim$ est le \emph{groupe libre} $\Fa$, où l'inverse de la classe d'équivalence de $x_1^{\epsilon_1}
    \cdots x_n^{\epsilon_n}$ est la classe d'équivalence de $x_n^{-\epsilon_n} \cdots x_1^{-\epsilon_1}$ avec
    $\epsilon_i \in \Z$ pour tout $i$. L'opération est la concaténation (la réduction est implicite).
  \end{defi}

  On fait souvent l'abus de language suivant: on va identifier un mot réduit avec sa classe d'équivalence.

  \begin{prop}
    \begin{enumerate}
    \item $\Fa$ est un groupe (von Dyck, 1882).
    \item La définition \ref{defi-1-grp-libre} est équivalente à la définition \ref{defi-2-grp-libre}.
    \end{enumerate}
  \end{prop}

  \begin{preuve}
    \begin{enumerate}
    \item
      \begin{itemize}
      \item Le neutre est le mot vide, noté $\epsilon$ ou $1_{\Fa}$.
      \item L'inverse de $a_1^{\epsilon_1} \cdots a_n^{\epsilon_n}$ est $a_n^{-\epsilon_n} \cdots
        a_1^{-\epsilon_1}$.
      \item L'opération de concaténation et réduction est associative (exercice)
      \end{itemize}

    \item Exercice. \qedhere
    \end{enumerate}
  \end{preuve}

  \paragraph{Question:} pourquoi dit-on  que $\Fa$ est libre sur $A$?
  \paragraph{Réponse:} car tout mot réduit sur $A$ représentant l'élément neutre est le mot vide
  (exercice). Alors il n'y a pas de relation entre les lettres dans $A$, et $\Fa$ à la présentation $\langle
  a_1, a_2, \ldots, a_n | - \rangle$.

  \section{Propriété universelle du groupe libre (PU)}
  \label{sec:propriete-universelle}
  
    Soit $G$ un groupe et $f:A \to G$ une application. Alors il existe un unique homomorphisme $\phi$ tel que
    le diagramme suivant commute.
    \begin{center}
      \begin{tikzpicture}
        \node (A) at (0,0) {$A$};
        \node (FA) at (3, 0) {$\Fa$};
        \node (G) at (1.5, -1.5) {$G$};
        \draw[right hook-latex] (A) to node[midway, above]{$i$} (FA);
        \draw[->, >=latex] (A) to node[midway, below left]{$f$} (G);
        \draw[->, >=latex] (FA) to node[midway, below right]{$!\phi$} (G);
      \end{tikzpicture}
    \end{center}
    Ceci signifie que toute application $f:A \to G$ s'étend en un unique homomorphisme $\phi: \Fa \to G$ où
    pour $w = a_{i_1}^{\epsilon_1} \cdots a_{i_n}^{\epsilon_n}$ on pose $\phi(w) = f(a_{i_1})^{\epsilon_1}
    \cdots f(a_{i_n})^{\epsilon_n}$ avec $\epsilon_i \in \Z$. En particulier, si $A$ est une partie
    génératrice de $G$ (par exemple $A = G$), on voit que $\Fa$ se surjecte sur $G$ et ceci nous donne le
    théorème suivant, qui est très important.

    \begin{theo} \label{theo-tt-grp-quotient-grp-libre}
      Tout groupe est quotient d'un groupe libre.
    \end{theo}

    \begin{preuve}
      Si $A$ est une partie génératrice d'un groupe $G$, par le premier théorème d'isomorphisme, il existe un
      isomorphisme tel que $\phi: \Fa \to G$ implique que $\Fa/\ker \phi \cong \im \phi = G$.
    \end{preuve}




  




%%% Local Variables:
%%% mode: latex
%%% TeX-master: "../GAC_cours.tex" 
%%% End:

% Chapter 2: Présentations des groupes
%---------------------------------------------------------%
%______//------             GAC             ------\\______%
%______||------         Chapitre 2          ------||______%
%______\\------  Présentations de groupes   ------//______%
%---------------------------------------------------------%

\chapter{Présentations de groupes}

  Soit $R \subset \Fa$. La \emph{fermeture normale $N(R)$} ou $\lhd R \rhd$ ou $gp_{\Fa}(R)$ dans $\Fa$ est
  définie par
    \[\bigcap_{\substack{N \lhd \Fa \\ R \subset N}}N.\]
  Il faut vérifier que 
  \begin{itemize}
  \item $N(R) \lhd \Fa$;
    
  \item $N(R) = \left\{ \prod_{r_{ij} \in R}w_{ij} r_{ij}^{\epsilon_j}w_{ij}^{-1} \right\}$ où $\epsilon_j =
    \pm 1$, $r_{ij} \in R$ et $w_{ij} \in \Fa$.
  \end{itemize}
  C'est en fait le plus petit sous-groupe normal contenant $R$.\\

  Si $G$ a une partie génératrice $A$, d'après la PU on a $G \cong \Fa/\ker \phi$ où $\phi: \Fa
  \xrightarrow{\mathrm{surj.}} G$. Alors si $\ker \phi = \lhd R \rhd$, on dit que $G$ est donné par la
  présentation $\langle A | R \rangle$. Les éléments de $A$ sont les \emph{générateurs} et les éléments de $R$
  sont les \emph{relateurs}.

  \begin{rems}
    \begin{enumerate}
    \item Si $|A|<+\infty$, on dit que $G$ est \emph{finiment engendré}.
    \item Si $|A|<+\infty$ et $|R|<+\infty$, on dit que $G$ est \emph{finiment présenté}.
    \end{enumerate}
  \end{rems}

  \begin{rems}
    \begin{enumerate}
    \item Si $S$ est un ensemble et $R \subset \F(S)$, la présentation $\langle S | R \rangle$ définit un
      \emph{unique groupe} (à isomorphisme près), le groupe $G = \F(S)/\lhd R \rhd$.

    \item Un groupe admet une infinité de présentations.
    \end{enumerate}
  \end{rems}

  \begin{exs}
    \begin{enumerate}
    \item Le groupe trivial: $T = \langle x | x = 1 \rangle$, $T = \langle a, b | a = b = 1 \rangle$.

    \item $(\Z^2, +) = \langle a, b | ab = ba \rangle$ où $a = (1,0)$ et $b=(0,1)$.
    \item $F_2 = \langle a,b | - \rangle$.
    \item $\Z/r\Z \times \Z/s\Z = C_r \times C_s = \langle x,y | x^r = 1, y^s = 1, xy = yx \rangle =
      \F(x,y)/\lhd x^r, y^s , [x,y] \rhd$ où $[x,y] = xyx^{-1}y^{-1} = 1$ est le commutateur.
    \item $G = \langle X | R \rangle$, $H = \langle Y | S \rangle$, $G \times H = \langle X \cup Y | R \cup S,
      xy = yx, x \in X, y \in Y \rangle$. \qedhere
    \end{enumerate}
  \end{exs}
  




%%% Local Variables:
%%% mode: latex
%%% TeX-master: "../GAC_cours.tex" 
%%% End:

% Chapter 3: Problèmes de Dehn
%---------------------------------------------------------%
%______//------             GAC             ------\\______%
%______||------         Chapitre 3          ------||______%
%______\\------     Problèmes de Dehn       ------//______%
%---------------------------------------------------------%

\chapter{Problèmes de Dehn}

  Supposons que $G$ soit donné par une présentation finie $\langle S | R\rangle$.
  \begin{itemize}
  \item[(1)] (PM) Problème des mots: soit $w \in \F(S)$. Est-ce que $w =_G 1$?
  \item[(1')] (PE) Problème de l'égalité des mots: soient $w_1, w_2 \in \F(S)$, est-ce que $w_1 =_G w_2 \iff
    w_1w_2^{-1} =_G 1$?
  \item[(2)] (PC) Problème de conjugaison: soient $w, v \in \F(S)$. Est-ce qu'il existe $g \in \F(S)$ tel que
    $g^{-1}wg =_G v$?
  \item[(3)] (PI) Problème de l'isomorphisme: soit $G_1 = \langle S_1|R_1 \rangle$ et $G_2 = \langle S_2 | R_2
    \rangle$ des présentations finies. Est-ce que $G_1 \cong G_2$?
  \end{itemize}

  La réponse à ces trois problèmes est qu'ils sont insolubles: il n'existe pas d'algorithme pour décider s'il
  y a une solution pour les trois questions (Adyan, Novikov-Boone, 1950-1960).

  \begin{ex}
    Soit $G = \langle x,y | x^2y^3 = x^3y^4 = 1\rangle$. On a que $x^3y^4 = 1 = x(x^2y^3)y = xy$, donc $x =
    y^{-1}$ et $y = x^{-1}$. Ainsi $x^2y^3 = x^2(x^{-1})^3 = x^{-1} = 1$, d'où $x = y = -1$. Ainsi $G$ est le
    groupe trivial!
  \end{ex}


  \begin{prop}
    Le problème des mots et le problème de conjugaison sont des invariants algébriques, ie pour deux
    présentations finies $\langle S_1|R_1\rangle, \langle S_2 |R_2\rangle$ d'un même groupe $G$, on a que les
    problème des mots pour $\langle S_1| R_1\rangle$ est résoluble ssi le problème des mots pour $\langle S_2
    | R_2\rangle$ est résoluble (pour PE aussi).
  \end{prop}

  \begin{preuve}
    Exercice. L'idée est que si on peut exprimer un mot dans $S_1$, on peut aussi l'exprimer dans $S_2$.
  \end{preuve}

  \section{Les problèmes de Dehn pour les groupes libres}
  \label{sec:pb-dehn-grps-libres}

    Soit $A = \{a, b, c, \ldots \}$, et $\Fa$ le groupe libre sur $A$. 
    \begin{enumerate}
    \item Problème des mots: soit $w =_{\Fa} 1 \iff $ après réductions, $w$ est le mot vide.

          $caa^{-1}b^{-2}b^2c^{-1} = 1$ (ou $\epsilon$) par réductions.

    \item[(1')] Problème d'égalité: $w_1, w_2$ deviennent $w_1', w_2'$ après réduction et on a que $w_1 =_{\Fa} w_2
      \iff w_1' \equiv w_2'$.
    \end{enumerate}

    \subsection{Problème de conjugaison pour les groupes libres}
    \label{sec:pb-conjugaison-grp-libre}

    \begin{defi}
      Si $w \in \Fa$ et $w = ava^{-1}$ avec $a \in A$ et $v \in \Fa$, l'opération $w
      \xrightarrow{\text{c. réd.}} v$ (enlever les $a$ et $a^{-1}$) s'appelle \emph{réduction cyclique de $w$}.
    \end{defi}

    \begin{ex}
      $w = a^{-1}bca^2b^{-1}a \xrightarrow{c.} bca^2b^{-1} \xrightarrow{c.} ca^2$.
    \end{ex}

    \begin{defi}
      Un mot $w$ est \emph{cycliquement réduit} s'il n'a pas une forme $w = ava^{-1}$, $a \in A, v \in \Fa$.
    \end{defi}

    \begin{defi}
      Deux mots $v, w$ sont \emph{conjugués cycliques} s'il existe des mots $\alpha$ et $\beta$ tels que $w =
      \alpha \beta$ et $v = \beta \alpha$.
    \end{defi}

    \begin{ex}
      $w = aab^{-1}c$. Un conjugué cyclique est $ab^{-1}ca$, en continuant on a $b^{-1}ca^2$, etc...
    \end{ex}

    L'algorithme pour résoudre le problème de conjugaison est le suivant. Soient $w_1$ et $w_2$ deux mots. On
    commence par faire la réduction cyclique des deux mots pour obtenir $w_1'$ et $w_2'$. $w_1'$ et $w_2'$
    sont donc cycliquement réduits. Si $w_1'$ et $w_2'$ sont conjugués cycliques, alors il existe $g$ tel que
    $gw_1g^{-1} = w_2$.

    \begin{ex}
      Soient $w_1 = cabc^{-1}$ et $w_2 = abbab^{-1}a^{-1}$. On effectue la réduction cyclique:
        \[w_1 \xrightarrow{c} ab, \quad w_2 \xrightarrow{c} bbab^{-1} \xrightarrow{c} ba.\]
      $ab$ et $ba$ sont conjugués cycliques, donc $w_1$ et $w_2$ sont conjugués. À la fin on obtient que 
        \[w_1 = (cab^{-1}a^{-1})w_2(cab^{-1}a^{-1})^{-1},\]
      ainsi $g = cab^{-1}a^{-1}$.
    \end{ex}

    \subsection{Problème de l'isomorphisme pours les groupes libres}
    \label{sec:pb-isom-grps-libres}

    Pour deux présentations $\langle X_1 | R_1 \rangle$ et $\langle X_2 | R_2 \rangle$, il n'y pas
    d'algorithme pour résoudre le problème de l'isomorphisme.


    Mais ici on sait qu'on a deux groupes libres.

    \begin{theo}
      Soient $X,Y$ deux ensembles (finis ou infinis). On a que $\F(X) \cong \F(Y) \iff |X| = |Y|$ ($|X| = |Y|$
      s'il y a une bijection $f:X \to Y$).
    \end{theo}

    \begin{preuve}
      \begin{description}
      \item["$\Rightarrow$":] Supposons qu'on ait une bijection $f: X \to Y$. Alors il existe $g = f^{-1}:Y
        \to X$. Par la propriété universelle, on a $\tilde{f}: X \to \F(Y)$, $i_X: X \hookrightarrow \F(X)$ et
        il existe un unique homomorphisme $\phi: \F(X) \to \F(Y)$. Même chose pour $Y$ on prend $\tilde{g}$,
        $i_Y$ et $\psi$.

        \begin{center}
          \begin{tikzcd}[column sep=small]
            X \arrow[rr, "\tilde{f}"] \arrow[rd, hookrightarrow, "i_X"] & & \F(Y) & &  
            X \arrow[rr, "\tilde{g}"] \arrow[rd, hookrightarrow, "i_Y"] & & \F(X) & &  
            X \arrow[rr, hookrightarrow, "i_X"] \arrow[rd, hookrightarrow, "i_X"] & & \F(X)  \\
             & \F(X) \arrow[ru, "\phi"] & & & & \F(Y) \arrow[ru, "\psi"] & & & & \F(X) \arrow{ru}[below right]{!\alpha =
            id_{\F(X)}} & 
          \end{tikzcd}
        \end{center}
        
        Alors $\psi \circ \phi : \F(X) \to \F(X)$ est une extension de $i_X$. Par l'unicité dans la propriété
        universelle, $\psi \circ \phi = id_{\F(X)}$.
        
        De même $\phi \circ \psi: \F(Y) \to \F(Y)$ est égal à $id_{\F(Y)}$. Donc $\phi$ et $\psi$ sont des
        isomorphismes et ainsi $\F(X) \cong \F(Y)$.

      \item["$\Leftarrow$":] Si $\F(X) \cong \F(Y)$, alors $|X| = |Y|$. Soit $N(X) = \langle g^2 | g \in \F(X)
        \rangle$. Montrons que $N(X)$ est un sous-groupe normal. La partie sous-groupe est claire, il reste
        donc à montrer qu'il est normal. $gh^2g^{-1} = (ghg^{-1})(ghg^{-1}) = (ghg^{-1})^2 \in N(x)$ (ce n'est
        pas la preuve complète, mais c'est l'idée). Ainsi $N(x) \lhd \F(X)$ et $\F(X)/N(X)$ est un groupe
        abélien, un 2-groupe, ie $x^2 = 1 \forall x \in \F(X)/N(X)$.
        \begin{enumerate}
        \item $(gN)^2 = gNgN = g^2N = N$ ce qui montre que c'est un 2-groupe.
        \item $(xy)^2 = 1 \implies xyxy = 1 \implies xy = y^{-1}x^{-1} = yx$ car les éléments sont d'ordre 2, ce
          qui montre que $\F(X)/N(X)$ est abélien.
        \end{enumerate}
        
        Notons $V(X) = \F(X)/N(X) = \underbrace{\Z/2\Z \oplus \cdots \oplus \Z/2\Z}_{|X|}$ car chaque élément
        engendre un groupe cyclique d'ordre 2. Ainsi $V$ est $\Z/2\Z$-espace vectoriel avec base $X$ et de
        dimension $|X|$.

        Comme $\F(X) \cong \F(Y)$ on a que $\F(X)/N(X) \cong \F(Y)/N(Y) \implies V(X) \cong V(Y) \implies
        |X|=|Y|$ car deux espaces vectoriels isomorphes ont des bases de mêmes cardinalités.
      \end{description}
    \end{preuve}

    

    

    
  




%%% Local Variables:
%%% mode: latex
%%% TeX-master: "../GAC_cours.tex" 
%%% End:

% Chapter 4: Propriétés du groupe libre
%----------------------------------------------------------%
%______//------             GAC              ------\\______%
%______||------         Chapitre 4           ------||______%
%______\\------  Propriétés du groupe libre  ------//______%
%----------------------------------------------------------%

\chapter{Propriétés du groupe libre}

  \begin{prop}
    Si $|A| \geq 2$, le centre de $\F(A)$ est trivial (ex), $Z(G) = \{g \in G | gh=hg \forall h \in G\}$.
  \end{prop}

  \begin{preuve}
    \begin{description}
    \item["$\supset$":] Cette inclusion est triviale, car l'élément neutre commute avec tout élément et ainsi
      $\{1\} \subset Z(\Fa)$.

    \item["$\subset$":] On va montrer la contraposée, c'est-à-dire que si $g \in \Fa$ avec $g \neq 1$, $g
      \notin Z(\Fa)$. Si $g = a_{i_1}^{\epsilon_1} \cdots a_{i_n}^{\epsilon_n}$, avec $\epsilon_i \neq
      -\epsilon_{n+1-i}$ pour tout $i$, et $\epsilon_1 \neq -\epsilon_{n-2}$ (pour qu'il n'y ait pas de réductions possible dans $g$). On pose
        \[h = a_{i_n}^{-\epsilon_n}a_{i_{n-1}}^{-\epsilon_{n-1}}a_{i_1}^{\epsilon_1} \cdots a_{i_{n-1}}^{\epsilon_{n-1}}.\]
      Ainsi, on a 
        \[gh = a_{i_1}^{\epsilon_1} \cdots a_{i_n}^{\epsilon_n} a_{i_n}^{-\epsilon_n}a_{i_{n-1}}^{-\epsilon_{n-1}}a_{i_1}^{\epsilon_1}
        \cdots a_{i_{n-2}}^{\epsilon_{n-2}} =  a_{i_1}^{\epsilon_1} \cdots a_{i_{n-2}}^{\epsilon_{n-2}}
        a_{i_1}^{\epsilon_1} \cdots a_{i_{n-2}}^{\epsilon_{n-2}} = (a_{i_1}^{\epsilon_1} \cdots
        a_{i_{n-2}}^{\epsilon_{n-2}})^2,\]
        \[hg = a_{i_n}^{-\epsilon_n}a_{i_{n-1}}^{-\epsilon_{n-1}}a_{i_1}^{\epsilon_1} \cdots a_{i_{n-2}}^{\epsilon_{n-2}}
        a_{i_1}^{\epsilon_1} \cdots a_{i_n}^{\epsilon_n}\]
      et $hg \neq gh$ car $h$ et $g$ sont irréductibles, et ne se réduisent quand on les multiplie car
      $\epsilon_1 \neq -\epsilon_{n-2}$ par hypothèse.
    \end{description}
  \end{preuve}

  \begin{prop}
    Si $|A| \geq 2$, $\F(A)$ est sans torsion (ex), (torsion: $\exists g \in G, n \geq 2 \in \N$ tq $g^n =
  1$).
  \end{prop}

  \begin{preuve}
    Exercice
  \end{preuve}

  \begin{theo}[de \textsc{Nielsen-Schreier}, 1927]
    Tout sous-groupe d'un groupe libre est libre.
  \end{theo}

  \begin{theo}[Version quantitative de Nielsen-Schreier]
    si $H$ est un sous-groupe d'indice $k$ de $\F_n$, alors $H \cong \F_{k(n-1)+1}$.
  \end{theo}

  \section{Observations}
  \label{sec:prop-grp-libre-obs}
  
  \begin{enumerate}
  \item $\F_2 \hookrightarrow \F_n$, $n \geq 2$. Par exemple $\F_2 = \langle a, b\rangle \hookrightarrow
    \langle a_1, a_2, \ldots, a_n \rangle$.
  \item L'autre direction ``fonctionne'' aussi, ie $\F_n \hookrightarrow F_2$, $n \geq 2$. Ainsi $\F_2$
    contient les groupes libres de rang $n$ pour chaque $n \in \N$.
  \end{enumerate}

  \begin{ex}
    Soit $\F_2 = \langle a,b \rangle$ et $\F_n = \langle a_1, a_2, \ldots, a_n \rangle$ et 
      \[f: \F_n \to \F_2, a_i \mapsto a^{-i}ba^i.\]
    Alors $f$ est un homomorphisme. On doit montrer que $f$ est injective, c'est-à-dire pour chaque mot réduit
    $a_{i_1}^{r_1} \ldots a_{i_m}^{r_m})$ dans $\F_n$ où $a_{i_j} \in \{a_1, \ldots, a_n \}$, $r_i \in \Z, i_j
    \neq i_{j+1}$. On
    va montrer que $f(a_{i_1}^{r_1} \ldots a_{i_m}^{r_m} \neq_{\F_2} 1$.

    On a que
      \[f(a_{i_1}^{r_1} \ldots a_{i_m}^{r_m}) = a^{-i_1}b^{r_1}a^{i_1}a^{-i_2}b^{r_2}a^{i_2} \cdots a^{i_m}
        \neq_{\F_2} 1\]
    car, par exemple, $i_1 \neq i_2$ ainsi il y a des réductions, mais ça ne se réduit pas au mot vide.
  \end{ex}

  \section{Groupes libres dans la nature}
  \label{sec:grp-libre-nature}
  
  Il y a des groupes libres partout!

  \begin{prop} \label{prop-ss-grp-sl2}
    Le sous-groupe de $SL_2(\Z)$ engendré par $l = \big( \begin{smallmatrix} 1&0\\ 2&1 \end{smallmatrix}\big)$
    et $r = \big( \begin{smallmatrix} 1&2\\ 0&1 \end{smallmatrix}\big)$ est libre de rang 2.
  \end{prop}

  La preuve utilise le Lemme du Ping-Pong.

  \begin{lem}[du Ping-Pong, Klein, 1880] \label{lem-ping-pong}
    Soit $G$ un groupe, $\alpha, \beta \in G$. On suppose que $G$ agit sur un ensemble $E$ ayant deux parties
    $X, Y \neq \varnothing$, tq $X \cap Y = \varnothing$ et
    \begin{itemize}
    \item $\forall m \in \Z \setminus \{0\}$, $\alpha^m \cdot y \in X$ pour tout $y \in Y$,
    \item $\forall m \in \Z \setminus \{0\}$, $\beta^m \cdot x \in Y$ pour tout $x \in X$.
    \end{itemize}
    Alors $\langle \alpha, \beta \rangle \cong \F_2$.
  \end{lem}

  \begin{figure}[h]
    \centering
    \begin{tikzpicture}
      \draw (0,0) rectangle (4, 3);
      \draw (2, 3) -- (2, 0);
      \draw (0.5, 3.5) node{$X$};
      \draw (3.5, 3.5) node{$Y$};
      \draw (4.5, 0.5) node{$E$};
      \node (x1) at (1, 2) {$\bullet$};
      \node (x2) at (3, 2) {$\bullet$};
      \node (y1) at (1, 1) {$\bullet$};
      \node (y2) at (3, 1) {$\bullet$};
      \draw (x1) node[left] {$x$};
      \draw (x2) node[right] {$\beta^p \cdot x$};
      \draw (y1) node[left] {$\alpha^m \cdot y$};
      \draw (y2) node[right] {$y$};
      \draw[->, >=latex] (x1) to [bend left] node [midway, above] {$\beta^p$} (x2);
      \draw[<-, >=latex] (y1) to [bend right] node [midway, above] {$\alpha^m$} (y2);
    \end{tikzpicture}
    \caption{Illustration du Lemme du Ping-Pong}
    \label{fig:lem-ping-pong}
  \end{figure}

  \begin{preuve}[du Lemme du Ping-Pong]
    Soit $m$ un mot réduit sur $\alpha, \beta$. $m$ est de la forme
    \begin{enumerate}
    \item $m = \alpha^{h_1} \beta^{k_1} \cdots \beta^{k_{n-1}} \alpha^{h_n}$ avec $h_i, k_i \in \Z \setminus
      \{0\}$.
        Alors supposons que $m =_G 1$. Ainsi $m \cdot Y = Y$.
          \[\alpha^{h_1} \cdots \beta^{k_{n-1}} \alpha^{h_n} \cdot Y \subseteq \alpha^{h_1} \cdots
          \beta^{k_{n-1}} \cdot X \subseteq \alpha^{h_1} \cdots \alpha^{k_{n-1}} \cdot Y \subset \cdots
          \subseteq \alpha^{h_1} \cdot Y\subset X,\]
        ainsi $m \cdot Y \subset X$, ce qui est une contradiction.

      \item $m = \beta^{k_1} \cdots \beta^{k_n}$, donc $\alpha^{-h_1}m\alpha^{h_1}$ est comme au point $1$ et
        ainsi $\alpha^{-h_1} m \alpha^{h_1} \neq_G 1$ ainsi $m \neq_G 1$.

      \item Si $m = \alpha^{h_1} \cdots \beta^{k_n}$, pour $h_0 \neq h_1$ on regarde
        $\alpha^{-h_0}(\alpha^{h_1} \cdots \beta^{k_n})\alpha^{h_0}$ qui est comme au point $1$. Donc $m
        \neq_G 1$

      \item $m = \beta^{k_1} \cdots \alpha^{h_n}$ et on fait la même preuve qu'au point 3.
    \end{enumerate}
    Ainsi $m \neq 1$ et $\langle \alpha, \beta \rangle \cong \F_2$.
  \end{preuve}

  \begin{preuve}[de \ref{prop-ss-grp-sl2}]
    Exercice.

    Début de la preuve: on regarde $E = \R^2$ et on regarde l'action de $SL_2(\Z)$ sur $\R^2$. 
      \[
      \begin{pmatrix}
        a & b \\ c & d
      \end{pmatrix}
      \cdot
      \begin{pmatrix}
        x \\ y
      \end{pmatrix} = 
      \begin{pmatrix}
        ax + by\\ cx + dy
      \end{pmatrix}
      \]

      \begin{center}
        \begin{tikzpicture}
          \draw[->, >=latex] (-3, 0) -- (4, 0);
          \draw[->, >=latex] (0, -1) -- (0, 4);
          \draw[color = OliveGreen] (-1, -1) -- (4, 4);
          \draw[color = OliveGreen] (2.5, 2.5) node[scale=0.8]{$\bullet$} node[below right]{$(x,y)$};
          \draw[dashed, color = OliveGreen] (-0.5, -1) -- (2, 4);
          \draw[color = OliveGreen] (1.5,3) node[scale=0.8]{$\bullet$} node[above left]{$(ax+by,cx+dy)$};
          \draw[->, >=latex, color = OliveGreen] (2.5, 2.5) to[bend right] (1.5, 3);
        \end{tikzpicture}
      \end{center}

      où $\cdot$ représente l'action (on prend simplement la multiplication). On ne va pas utiliser seulement
      des points, mais des droites vectorielles. On prend la droite qui passe par l'origine et $(x,y) \in
      \R^2$ et on regarde l'image de cette droite par l'action, qui est aussi une droite vectorielle. On peut
      considérer l'action sur l'ensemble des droites vectorielles dans $\R^2$ qui est l'espace projectif de
      dimension 1, $\mathbb{P}\R^1$ (une droite projective peut être vue comme ``demi-cercle'' où $A = B$).
      Il faut donc montrer que $X$ et $Y$ satisfont l'hypothèse du Lemme du Ping-Pong. 

      \begin{center}
        \begin{tikzpicture}
          \draw (-5, 0) -- (5, 0);
          \draw (3, 0) arc(0:180:3);
          \draw (-3, 0) node[below]{$A$};
          \draw (3, 0) node[below]{$B$};
          \draw (0,0) node[above]{$O$};
          \draw (-4, 4) -- (0,0) -- (4,4);
          \draw (-3, 1) node[above left]{$Y$};
          \draw (3, 1) node[above right]{$Y$};
          \draw (0, 3) node[above] {$X$};
          \draw (4,4) node[above right]{$\mathbb{P}\R^1$};
        \end{tikzpicture}
      \end{center}

  \end{preuve}

  \begin{rem}
    On trouve des groupes libres très souvent dans les groupes linéaires.
  \end{rem}

  \begin{theo}[``Alternative de Tits'', 1971]
    Soit $G$ un groupe linéaire, c'est-à-dire un sous-groupe de $GL_n(\C)$ pour un certain $n \geq 1$. On a
    l'alternative:
    \begin{itemize}
    \item ou bien $G$ est virtuellement résoluble;
    \item ou bien $G$ contient $\F_2$ comme sous-groupe.
    \end{itemize}
  \end{theo}

  \begin{ex}
    Considérons $\mathrm{Homeo}(\R) = \{\phi: \R \to \R\, |\, \phi \text{est continue et
      bijective}\}$. $\mathrm{Homeo}(\R)$ contient beaucoup de groupes libres. 
    \begin{equation*}
      \begin{cases}
        f(x) &= x^p, \text{ $p$ premier impair,}\\
        g(x) &= x+1.
      \end{cases}
    \end{equation*}
    Alors $\langle f(x), g(x)\rangle \cong \F_2 $ (la preuve est très difficile).
  \end{ex}
  
  

  




%%% Local Variables:
%%% mode: latex
%%% TeX-master: "../GAC_cours.tex" 
%%% End:

% Chapter 5: Introduction à la topologie algébrique
%--------------------------------------------------------%
%______//------             GAC            ------\\______%
%______||------         Chapitre 5         ------||______%
%______\\------ Intro à la topo algébrique ------//______%
%--------------------------------------------------------%

\chapter{Introduction à la topologie algébrique}

  À tout espace topologique $X$ raisonnable, on associe des groupes.

  Une propriété fondamentale est qu'à toute application continue $f: X \to Y$ correspond un homomorphisme de
  groupes $f_\ast:F(X) \to F(Y)$.

  \section{Groupe fondamental d'un espace topologique}
  \label{sec:grp-fondamental-esp-topo}

    \subsection{Lacets}

    \begin{defi}
      Soit $X$ un espace topologique. Un \emph{arc} dans $X$ est une application continue $\gamma: [0,1] \to
      X,\, t \mapsto \gamma(t)$, où $\gamma(0)$ est l'\emph{origine} de $\gamma$ et $\gamma(1)$ est
      l'\emph{extrémité} de $\gamma$.
    \end{defi}

    
    Un arc peut être inversé:
      \[\check{\gamma}(t) = \gamma(1-t).\]
    Deux arcs $\gamma, \delta$ peuvent être composés si l'origine de $\delta$ est l'extrémité de $\gamma$. 
      \[(\gamma\delta)(t) =
      \begin{cases}
        \gamma(2t) & \text{si } 0 \leq t \leq \frac{1}{2},\\
        \delta(2t-1) & \text{si } \frac{1}{2} \leq t \leq 1.
      \end{cases}
      \]
    Pour avoir une composition toujours bien définie, on se restreint aux \emph{lacets}, c'est-à-dire les arcs
    tels que $\gamma(0) = \gamma(1) = x_0$. Si $x_0 = \gamma(0) = \gamma(1)$, on dit que $\gamma$ est
    \emph{basée en $x_0$}.

    En 1901, \textsc{Poincaré} (1854-1912) a eu l'idée que, si on regarde les lacets à déformation continue
    près, on obtient un groupe, qui détecte la présence de ``trous'' dans $X$.

    \begin{defi}
      Soient $\gamma_0$, $\gamma_1$ deux lacets basés en $x_0$. Une \emph{homotopie} de $\gamma_0$ à
      $\gamma_1$ est une application continue
        \[
          F:  [0,1] \times [0,1] \to X\\
        \]
      telle que
        \[
        \begin{cases}
          F(0,t) = \gamma_0(t), & \forall t \in [0,1],\\
          F(s,0) = F(s,1) = x_0, & \forall s \in [0,1], \\
          F(1,t) = \gamma_1(t), & \forall t \in [0,1].
        \end{cases}
        \]
    \end{defi}

    \begin{figure}[h]
      \centering
      \begin{tikzpicture}
        \draw (0,0) node[scale=0.8]{$\bullet$} node[below]{$x_0$};
        \draw[color = OliveGreen] plot[domain={-pi/2}:{pi/2}, samples = 80] ({((2*cos(\x r))/(1+sin(\x r)*sin(\x r)))},%
                                                        {((2*sin(\x r)*cos(\x r)/(1+sin(\x r)*sin(\x r))))});
        \draw[color = OliveGreen, dashed] plot[domain={-pi/2}:{pi/2}, samples = 80] ({((3*cos(\x r))/(1+sin(\x r)*sin(\x r)))},%
                                                        {((3*sin(\x r)*cos(\x r)/(1+sin(\x r)*sin(\x r))))});
        \draw[color = OliveGreen] plot[domain={-pi/2}:{pi/2}, samples = 80] ({((4*cos(\x r))/(1+sin(\x r)*sin(\x r)))},%
                                                        {((4*sin(\x r)*cos(\x r)/(1+sin(\x r)*sin(\x r))))});
        \draw (2, 0) node[color = OliveGreen, left, scale = 0.8]{$\gamma_0$};
        \draw (3, 0) node[color = OliveGreen, left, scale = 0.8]{$\gamma_s$};
        \draw (4, 0) node[color = OliveGreen, left, scale = 0.8]{$\gamma_1$};
      \end{tikzpicture}
      \caption{Exemple d'homotopie}
      \label{fig:exemple-homotopie}
    \end{figure}

    Si on pose $\gamma_s(t) = F(s,t)$, on voit que $(\gamma_s)_{s \in [0,1]}$ est une famille continue de
    lacets qui interpole entre $\gamma_0$ et $\gamma_1$.

    \begin{defi}
      Deux lacets $\gamma_0$ et $\gamma_1$ (basés en $x_0$) sont \emph{homotopes} s'il existe une homotopie de
      $\gamma_0$ à $\gamma_1$, et dans ce cas on écrit $\gamma_0 \sim \gamma_1$. On écrit le lacet trivial
      basé en $x_0$ $\epsilon_{x_0}$. Si $\gamma \sim \epsilon_{x_0}$, on dit que $\gamma$ est homotope à zéro.
    \end{defi}

    \begin{prop}
      Pour les lacets basés en $x_0 \in X$, la relation ``être homotope'' est une relation d'équivalence. On
      note $[\gamma]$ la classe d'équivalence de $\gamma$.
    \end{prop}

    \begin{preuve}
      Exercice.
    \end{preuve}

    \begin{figure}[h]
      \centering
      \begin{tikzpicture}
        \draw (0,0) node[scale=0.8]{$\bullet$} node[below]{$x_0$};
        \begin{scope}[rotate=30]
        \draw[color = OliveGreen] plot[domain={pi/2}:{3*pi/2}, samples = 80] ({((4*cos(\x r))/(1+sin(\x r)*sin(\x r)))},%
                                                        {((4*sin(\x r)*cos(\x r)/(1+sin(\x r)*sin(\x r))))});
        \end{scope}
        \begin{scope}[rotate=40]
        \draw[color = OliveGreen] plot[domain={-pi/2}:{pi/2}, samples = 80] ({((2*cos(\x r))/(1+sin(\x r)*sin(\x r)))},%
                                                        {((2*sin(\x r)*cos(\x r)/(1+sin(\x r)*sin(\x r))))});
        \end{scope}
        \draw (-5, 3) rectangle (3, -4);
        \draw (-5, 3) node[below right] {$X$};
        \draw (-2, -1.5) circle(0.5);
      \end{tikzpicture}
      \caption{Exemple d'homotopies ayant des classes d'équivalence différentes (le rond est un ``trou'')}
      \label{fig:exemple-homotopie-classes-equiv}
    \end{figure}

    
    \subsection{Groupe fondamental}

    \begin{theo}[-définition] \index{Groupe!Fondamental}
      On note $\Pi_1(X, x_o)$ l'ensemble des classes d'homotopie des lacets de $X$ basés en $x_o$. Avec la
      multiplication $[\gamma][\delta] = [\gamma \delta]$, $\Pi_1(X, x_0)$ est un groupe, appelé \emph{groupe
        fondamental} de $X$ (en $x_0$).

      L'élément neutre est $[\epsilon_{x_0}]$ et l'inverse de $[\gamma]$ est $[\check{\gamma}]$.
    \end{theo}

    \begin{preuve}
      On vérifie d'abord que, si $\gamma_0 \sim \gamma_1$, $\delta_0 \sim \delta_1$ alors $\gamma_0\delta_0
      \sim \gamma_1\delta_1$, c'est-à-dire que la multiplication est bien définie. On a donc que $[\gamma_0] =
      [\gamma_1]$ et $[\delta_0 = \delta_1] \Rightarrow [\gamma_0 \delta_0] = [\gamma_1 \delta_1]$. 

      Soient $F$ et $G$ deux homotopies de $\gamma_0$ à $\gamma_1$ et de $\delta_0$ à $\delta_1$
      respectivement. Une homotopie de $\gamma_0\delta_0$ à $\gamma_1\delta_1$ est donnée par 
        \[
        H(s,t) = 
        \begin{cases}
          F(s,2t) & \text{si } 0 \leq t \leq \frac{1}{2},\ s \in [0,1]\\
          G(s, 2t-1) & \text{si } \frac{1}{2} \leq t \leq 1,\ s \in [0,1]
        \end{cases}
        \]
      (à vérifier).

      Il faut encore montrer que:
      \begin{itemize}
      \item $\epsilon_{x_o}\gamma \sim \gamma \sim \gamma \epsilon_{x_0}$;
      \item $\gamma \check{\gamma} \sim \epsilon_{x_0}$;
      \item associativité: si $\gamma_0$, $\gamma_1$, $\gamma_2$ sont trois lacets basés en $x_0$,
        $\gamma_0(\gamma_1\gamma_2) \sim (\gamma_0\gamma_1)\gamma_2$.
      \end{itemize}
    \end{preuve}



    \subsection{Propriétés du groupe fondamental}

    \paragraph{Rappel:} Un espace est \emph{connexe par arcs} si deux points peuvent être joints par un arc.

    \begin{prop}\index{Connexe!par arcs}
      Si $X$ est connexe par arc, alors 
        \[\Pi_1(X, x_0) \cong \Pi_1(X, y_0)\ \forall x_0, y_0 \in X.\]
    \end{prop}
    
    \begin{conseq}
      Si $X$ est connexe par arcs, on peut parler du \emph{groupe fondamental de $X$}, noté $\Pi_1(X)$.
    \end{conseq}


    \begin{preuve}
      Exercice. Dessin de l'idée de la preuve:
      \begin{center}
        \begin{tikzpicture}
          \draw (-2, 2) rectangle (5, -1);
          \draw (5, 2) node[below left]{$X$};
          \draw (0,0) node[scale=0.8]{$\bullet$} node[below]{$x_0$};
          \draw (4,0) node[scale=0.8]{$\bullet$} node[below]{$y_0$};
          \begin{scope}[rotate=-50]
            \draw[color = OliveGreen, ->, >=latex] plot[domain={pi/2}:{3*pi/2}, samples = 80] ({((2*cos(\x
              r))/(1+sin(\x r)*sin(\x r)))}, {((2*sin(\x r)*cos(\x r)/(1+sin(\x r)*sin(\x r))))});
          \end{scope}
            \draw[color = OliveGreen](-1, 2) node[below, scale=0.8]{$\gamma$};
          \draw[color = OliveGreen] (0,0) to[bend left] node[midway, above, scale = 0.8]{$\alpha$} (2, 0)
          to[bend right] (4,0);
        \end{tikzpicture}
      \end{center}
      Ainsi pour passer de $\gamma \in \Pi_1(X, x_0)$ à un élément de $\Pi_1(X, y_0)$, on prend
      $\check{\alpha}\gamma \alpha$.
    \end{preuve}
    
    \begin{defi} \index{Connexe!simplement}
      Un espace $X$ (connexe par arcs) est \emph{simplement connexe} si $\Pi_1(X) = 0$ (ou
      $\{1\}$). C'est-à-dire que tout lacet dans $X$ est homotope à $\epsilon_{x_0}$.
    \end{defi}
    

    \begin{exs}
      \begin{enumerate}
      \item Un tel ensemble de $\R^n$ est simplement connexe:
        \begin{center}
        \begin{tikzpicture}
          \draw (0,0) node[scale=0.8]{$\bullet$} node[below]{$x_0$};
          \begin{scope}[rotate=40]
            \draw[color = OliveGreen, ->, >=latex] plot[domain={pi/2}:{3*pi/2}, samples = 80] ({((2*cos(\x
              r))/(1+sin(\x r)*sin(\x r)))}, {((2*sin(\x r)*cos(\x r)/(1+sin(\x r)*sin(\x r))))});
          \end{scope}
            \draw (2,0) -- (3, -2) -- (-1, -2) -- (-3, -1) -- (-2, 2) -- (0, 1) -- (1, 2) -- cycle;
        \end{tikzpicture}
      \end{center}
      \item Les arbres sont simplements connexes:
        \begin{center}
          \begin{tikzpicture}
            \draw (-1, -1) -- (0,0) -- (1, -1) -- (0,0) -- (1.5, 2) -- (2.5, 1) -- (1.5, 2) -- (2, 3) -- (1,
            4) -- (2, 3) -- (3, 4);
            \draw[color = OliveGreen!80] (0,0) node[scale = 0.8]{$\bullet$};
            \draw[color = OliveGreen!80] (-1,-1) node[scale = 0.8]{$\bullet$};
            \draw[color = OliveGreen!80] (1,-1) node[scale = 0.8]{$\bullet$};
            \draw[color = OliveGreen!80] (1.5,2) node[scale = 0.8]{$\bullet$};
            \draw[color = OliveGreen!80] (2.5,1) node[scale = 0.8]{$\bullet$};
            \draw[color = OliveGreen!80] (2,3) node[scale = 0.8]{$\bullet$};
            \draw[color = OliveGreen!80] (1,4) node[scale = 0.8]{$\bullet$};
            \draw[color = OliveGreen!80] (3,4) node[scale = 0.8]{$\bullet$};
          \end{tikzpicture}
        \end{center}
      \item L'ensemble suivant est homéomorphe à $[0,1] \times [0,1]$.
        \begin{center}
          \begin{tikzpicture}
            \draw (0,0) arc(90:270:2);
            \draw (0,-1) arc(90:270:1);
            \draw (0, 0) arc(90:-90:0.5); 
            \draw (0, -3) arc(90:-90:0.5);
            \filldraw[color = OliveGreen!80, opacity = 0.15] (0,0) arc(90:270:2) arc(-90:90:0.5) 
            arc(270:90:1) arc(-90:90:0.5);
          \end{tikzpicture}
        \end{center}
      \item Pour $n \geq 2$, la sphère $\mathbb{S}^{n}$ est simplement connexe ($S^1$ n'est pas simplement connexe).
      \end{enumerate}
    \end{exs}

    \begin{prop}
      Soit $f: X \to Y$ une application continue, avec $y_0 = f(x_0)$. On pose $f_\ast: \Pi_1(X, x_0) \to
      \Pi_1(Y, y_0), [\gamma] \mapsto [f \circ \gamma]$. Alors $f_\ast$ est un homomorphisme de groupes.

      De plus, 
      \begin{enumerate}
      \item si $f:X \to Y$, $g: Y \to Z$ sont continues avec $y_0 = f(x_0)$ et $z_0 = g(y_0)$, alors $(g \circ
        f)_\ast = g_\ast \circ f_\ast$;
      \item $id_X : X \to X$, alors $(id_X)_\ast = Id_{\Pi_1(X, x_0)}$.
      \end{enumerate}
    \end{prop}

    \begin{preuve}
      Exercice.
    \end{preuve}

    \begin{theo}
      On a que
        \[\Pi_1(S^1) \cong \Z\]
    \end{theo}

    \begin{preuve}
      Difficile, et long.
    \end{preuve}

    \begin{exs}
      \begin{enumerate}
      \item Soit $\Pi^2$ le tore. En découpant le long de $a_1$ et $a_2$, on obtient un carré. Ceci montre que
        $[a_1a_2a_1^{-1}a_2^{-1}] = 1$ dans $\Pi_1(\Pi^2)$. Ainsi $\Pi_1(\Pi^2) = \Z^2$.

        Si on enlève à $\Pi^2$ un petit disque ouvert $D$, le bord de $D$ est $a_1a_2a_1^{-1}a_2^{-1}$ dans
        $\Pi_1(X)$, où $X = \Pi^2 \setminus D$. En fait, $\Pi_1(X) \cong \F_2 = \langle a_1, a_2 \rangle$
        ($\F_2$ est le groupe libre). 

        
        \begin{center}
          \begin{tikzpicture}
            \draw plot[domain = 0:2*pi, samples = 80]({5*cos(\x r)}, {3*sin(\x r)});
            \draw plot[domain = pi:2*pi, samples = 80]({3*cos(\x r)}, {1*sin(\x r)+0.6});
            \draw plot[domain = 0:pi, samples = 80]({2*cos(\x r)}, {0.5*sin(\x r)-0.15});
            % % TO ADD: cercles générateurs du tore + petit cercle à enlever
          \end{tikzpicture}
        \end{center}

      \item On a que $\Pi_1(\Sigma_2) = \langle a_1, a_2, b_1, b_2 | [a_1, a_2][b_1, b_2] = 1 \rangle$.
        \begin{center}
          \begin{tikzpicture}
            % TODO: Dessiner double tore
          \end{tikzpicture}
        \end{center}
      \end{enumerate}
    \end{exs}

  \section{Produits libres}
  \label{sec:produits-libres}

  \begin{defi} \index{Produit libre}
    Soient $A$ et $B$ deux groupes. Le \emph{produit libre}, noté $G = A \ast B$ est l'ensemble des mots de la
    forme
      \[a_1b_1a_2b_2 \cdots a_kb_k,\ k \in \N,\ a_i \in A,\ b_i \in B\]
    et $a_2, \ldots, a_k \neq \epsilon_A$ et $b_1, \ldots, b_{k-1} \neq \epsilon_B$. 
  \end{defi}

  Donc $G$ est l'ensemble des mots obtenus en alternant un élément non trivial d'un groupe, un élément non
  trivial de l'autre, etc.

  \begin{ex}
    \begin{enumerate}
    \item $\Z \ast \Z = \F_2 = \langle a, b \rangle$.

    \item En général, $\F_k \ast \F_m \cong \F_{k+m}$.

    \item Soit $D_\infty$ le groupe dihédral infini, c'est le sous-groupe des isométries de $\R$ engendré par
      deux symétries centrales. Alors
        \[D_\infty \cong \Z/2\Z \ast \Z/2\Z\]
      où $\Z/2\Z = \langle s_1 \rangle$ et $\Z/2\Z = \langle s_2 \rangle$. En effet, prenons
      \begin{center}
        \begin{tikzpicture}
          \draw[->, >=latex] (-5, 0) -- (5,0) node[right]{$\R$};
          \draw (-2, 0) node[scale=0.8]{$\bullet$} node[above left]{$-1$};
          \draw (2, 0) node[scale=0.8]{$\bullet$} node[above right]{$1$};
          \draw (0,0.2) -- (0, -0.2);
          \draw (0,0) node[above right]{$0$};
          \draw[dashed, color = OliveGreen!80] (-2, 1) -- (-2, -1);
          \draw[dashed, color = OliveGreen!80] (2, 1) -- (2, -1);
          \draw[<->, >=latex, color = OliveGreen!80] (-3, 0) to[bend right] node[midway, below left]{$s_1$} (-1, 0);
          \draw[<->, >=latex, color = OliveGreen!80] (1, 0) to[bend right] node[midway, below right]{$s_2$} (3, 0);
        \end{tikzpicture}
      \end{center}
      On voit que pour tout $s_{i_1} \cdots s_{i_k}$ avec $i_j \neq i_{j+1}$, on a $s_{i_1} \cdots s_{i_k}
      \neq 0$, ainsi $s_{i_1} \cdots s_{i_k} \neq \epsilon_{D_\infty}$.
    \end{enumerate}
  \end{ex}

  \begin{lem}[du Ping-Pong, 2ème version] \index{Lemme!du Ping-Pong!2nde version}
    Soient $G_1, G_2$ des sous-groupes de $Sym(X)$. On suppose que $|G_1| \geq 2$, $|G_2| \geq 3$. S'il existe
    deux parties $A_1, A_2 \subset X$ telles que $A_i \neq \varnothing$, $A_1 \not\subset A_2$ avec 
    \begin{itemize}
    \item $g_1(A_1) \subseteq A_2 \forall g_1 \in G_1 \setminus \{id\}$;
    \item $g_2(A_2) \subseteq A_1 \forall g_2 \in G_2 \setminus \{id\}$,
    \end{itemize}
    alors le sous-groupe engendré par $G_1 \cup G_2$ dans $Sym(X)$ est isomorphe à $G_1 \ast G_2$.
  \end{lem}

  \section{Théorème de Van Kampen (version simple)}

  \begin{theo}[de Van Kampen] \index{Théorème!de Van Kampen}
    Soit $X$ un espace connexe par arcs. On suppose que $X = U \cup V$ où 
    \begin{itemize}
    \item $U$ et $V$ sont des ouverts connexes par arcs;
    \item $U \cap V$ est simplement connexe et non vide.
    \end{itemize}
    Alors $\Pi_1(X) \cong \Pi_1(U) \ast \Pi_1(V)$ (produit libre des groupes fondamentaux).
  \end{theo}

  \begin{exs} \index{Bouquet!à deux cercles} \index{Bouquet!à $n$ cercles}
    \begin{enumerate}
    \item Le bouquet à deux cercles. Si $X = U \cup V$, on a $U \cap V = \{x\}$.
      \begin{center}
        \begin{tikzpicture}
          \draw (0,0) node[scale=0.8]{$\bullet$} node[above]{$x$};
          \draw[color = OliveGreen, ->, >=latex] 
               plot[domain={pi/2}:{3*pi/2}, samples = 80] 
                   ({((2*cos(\x r))/(1+sin(\x r)*sin(\x r)))}, {((2*sin(\x r)*cos(\x r)/(1+sin(\x r)*sin(\x r))))});
           \draw[color = RoyalBlue, ->, >=latex] 
               plot[domain={-pi/2}:{pi/2}, samples = 80] 
                   ({((2*cos(\x r))/(1+sin(\x r)*sin(\x r)))}, {((2*sin(\x r)*cos(\x r)/(1+sin(\x r)*sin(\x r))))});
           \draw[color = OliveGreen!80] (-2, 0) node[right]{$U$};
           \draw[color = RoyalBlue] (2, 0) node[left]{$V$};
        \end{tikzpicture}
      \end{center}
      Le groupe fondamental est
        \[\Pi_1(X) = \Pi_1(U) \ast \Pi_1(V) = \Z \ast \Z = \F_2.\]

      \item Si $(X, x_o)$ et $(Y, Y_0)$ sont deux espaces pointés (car on a donné des points), le
        \emph{wedge} ou \emph{joint} de $X$ et $Y$ est $X \wedge Y = X \cupdot Y / x_0=y_0$. Si $x_0, y_0$
        possèdent des voisinages simplement connexes, alors
          \[\Pi_1(X \wedge Y) = \Pi_1(X, x_0) \ast \Pi_1(Y, y_0).\]
        Par exemple si on prend $X = S^1$ et $Y = \Pi^2$, on obtient la chose suivante pour $X \wedge Y$.

        \begin{center}
          \begin{tikzpicture}
            \draw plot[domain = 0:2*pi, samples = 80]({5*cos(\x r)}, {3*sin(\x r)});
            \draw plot[domain = pi:2*pi, samples = 80]({3*cos(\x r)}, {1*sin(\x r)+0.6});
            \draw plot[domain = 0:pi, samples = 80]({2*cos(\x r)}, {0.5*sin(\x r)-0.15});
            \draw[color = OliveGreen] (-3, -1) node[scale = 0.8]{$\bullet$} node[below right]{$x_0 = y_0$};
            \draw[color=OliveGreen] (-3, 0) circle(1);
          \end{tikzpicture}
        \end{center}

      \item On appelle $B_n$ le bouquet de $n$ cercles. Alors
          \[\Pi_1(B_n) = \F_n = \langle a_1, \ldots, a_n\rangle. \]
        Plus généralement, si $X = (V,E)$ et un graphe connexe avec $n = |V|$, $m = |E|$ vu comme espace
        topologique en identifiant chaque arête à une copie de $[0,1]$, alors
          \[\Pi_1(X) \cong \F_{m-n+1}.\]
        Par exemple si $G$ est le graphe suivant:
        \begin{center}
          \begin{tikzpicture}
            \node[scale=0.8] (a) at (0,0) {$\bullet$};
            \node[scale=0.8] (b) at (1,0) {$\bullet$};
            \node[scale=0.8] (c) at (0,-1) {$\bullet$};
            \node[scale=0.8] (d) at (1,-1) {$\bullet$};
            \draw (a.center) -- (b.center) -- (c.center) -- (d.center) -- (a.center) -- (c.center);
            \draw (b.center) -- (d.center);
            \draw (a.center) -- (b.center) -- (c.center) -- (d.center) -- (a.center) -- (c.center);
            \draw (b.center) -- (d.center);
            \draw[color=OliveGreen, line width=1.5pt, opacity = 0.1] (c.center) -- (a.center) -- (b.center) --
            (a.center) -- (d.center);
          \end{tikzpicture}
        \end{center}
          \[\Pi_1(G) = \F_{6-4+1} = \F_3.\]
        En effet, soit $\mathcal{T}$ un \emph{arbre maximal} \index{Arbre!maximal} de $X$ (un \emph{arbre
          maximal} est un sous-graphe de $X$, sans circuit passant par tous les sommets). En contractant
        $\mathcal{T}$ sur un point, on obtient un bouquet à $m-n+1$ cercles, car $\mathcal{T}$ a $n-1$ arêtes.
        \begin{center}
          \begin{tikzpicture}
            \node[scale=0.8] (a) at (0,0) {$\bullet$};
            \node[scale=0.8] (b) at (1,0) {$\bullet$};
            \node[scale=0.8] (c) at (0,-1) {$\bullet$};
            \node[scale=0.8] (d) at (1,-1) {$\bullet$};
            \draw (a.center) -- (b.center) -- (c.center) -- (d.center) -- (a.center) -- (c.center);
            \draw (b.center) -- (d.center);
            \draw[color=OliveGreen, line width=1.5pt, opacity = 0.1] (c.center) -- (a.center) -- (b.center) --
            (a.center) -- (d.center);
            \begin{scope}[xshift = 3cm]
               \node[scale=0.8] (a) at (0,0) {$\bullet$};
               \node[scale=0.8] (b) at (1,0) {$\bullet$};
               \node[scale=0.8] (c) at (0,-1) {$\bullet$};
               \node[scale=0.8] (d) at (1,-1) {$\bullet$};
               \draw (a.center) -- (b.center) -- (c.center) -- (d.center) -- (a.center) -- (c.center);
               \draw (b.center) -- (d.center);
               \draw[color=OliveGreen, line width=1.5pt, opacity = 0.1] (a.center) -- (b.center) -- (c.center)
               -- (d.center);
            \end{scope}
          \end{tikzpicture}
        \end{center}
        Ci-dessus on a des arbres maximaux, car il reste 3 arêtes quand on contracte les arêtes vertes (on
        obtient donc un bouquet à 3 arêtes, dont le $\Pi_1$ est $\F_3$).
    \end{enumerate}
  \end{exs}




  \section{Revêtements}
  \label{sec:revetements}


  Tous les espaces sont supposés connexes par arcs et localement connexes par arcs.

  \begin{defi} \index{Revêtement}
    Un triplet $(X,Y,p)$, noté $\substack{Y\\\downarrow\\ X}p$ est un \emph{revêtement} de $X$ si: 
    \begin{itemize}
    \item $p$ est une application continue surjective $Y \to X$.
    \item Pour tout $x \in X$, $p^{-1}(X)$ est discret dans $Y$.
    \item Tout $x \in X$ possède un \emph{voisinage trivialisant} \index{Voisinage!trivialisant} $U_x$,
      c'est-à-dire un voisinage connexe par arcs tel que $p^{-1}(U_x)$ est homéomorphe à $p^{-1}(x) \times
      U_X$, par un homéomorphisme $h_x: p^{-1}(U_x) \to p^{-1} \times U_x$ tel que le diagramme suivant
      commute (où $p_2$ est la projection sur le 2ème facteur).
      \begin{center}
        \begin{tikzcd}[column sep = small]
          p^{-1}(U_x) \arrow[rr, "h_x"] \arrow[rd, "p \big|_{p^{-1}(U_x)}" below left] & & p^{-1}(x) \times U_x \\
          & U_x \arrow[ru, leftarrow, "p_2" below right] &
        \end{tikzcd}
      \end{center}
    \end{itemize}
    L'image mentale d'un revêtement est celle de la ``pile d'assiettes''.

    \begin{center}
      \begin{tikzpicture}
        \draw (-5, 0) -- (5, 0) node[right]{$X$};
        \draw (0,0) node[scale=0.8]{$\bullet$};
        \draw[color = OliveGreen, line width = 1.5pt, opacity = 0.1] (-1,0) to node[near end, below]{$U_x$} (1,0);
        \draw (-3.5, 3) arc(225:315:5) node[right]{$Y$};
        \foreach \x in {1,2,...,5} {
          \draw[color = RoyalBlue] (0,{1.5+(\x/2)}) node[scale=0.8]{$\bullet$};
          \draw[color = OliveGreen, line width = 1.5pt, opacity = 0.1] (-1,{1.5+(\x/2)}) -- (1,{1.5+(\x/2)});
        }
        \draw[color=RoyalBlue] (0, 2) node[below]{$p^{-1}(x)$};
        \draw[color=OliveGreen] (-1, 4) node[above]{$p^{-1}(U_x)$};
      \end{tikzpicture}
    \end{center}

    On dira que $\substack{Y\\\downarrow\\ X}p$ est un \emph{revêtement à $n$ feuillets} \index{Revêtement!à
      $n$ feuillets} si $\# p^{-1}(x) = n$, et a \emph{une infinité de feuillets} si $\# p^{-1}(x) = \infty$.
  \end{defi}

  
  \begin{exs}
    \begin{enumerate}
    \item Soit $X = \{z \in \C\ |\ |z| = 1 \}$. Alors l'espace $Y$ peut être représenté par une hélice, mais $Y
      = \R$. Alors $p: \R \to \S^1$ est défini par $p(t) = e^{2\pi i t}$ et $\substack{Y\\\downarrow\\ X}p$
      est un revêtement car $p$ est surjective. Si $z = e^{2\pi i \phi}$, alors $p^{-1}(z) = \phi + \Z$ est
      discret dans $\R$. Enfin, si $z = e^{2 \pi i \phi} \in S^1$, $U_z = S^1 \setminus \{-z\}$ (tout le
      cercle sauf le point opposé à $z$) est un voisinage de $z$, et $p^{-1}(U_z) = \R \setminus \{\phi +
      (2k+1)\pi, k \in \Z\} \cong \Z \times U_z$ car $\Z = p^{-1}(z)$. On ne prend pas les multiples impairs
      de $\pi$ car on ne veut pas $z+\pi, z-\pi, z+3\pi, \ldots$ dans le revêtement.

    \item Soit $X = S^1$, $Y = S^1$ et $p: S^1 \to S^1,\ z \mapsto z^n$ avec $n > 0$ et un revêtement à $n$
      feuillets. On parcourt le cercle $n$ fois, et on arête au même point qu'on a commencé.

    \item Si $x$ est un bouquet à deux boucles, alors $Y_{1,n}$ défini comme suit est un revêtement à $n$
      feuillets. $Y_{1, \infty}$ a une infinité de feuillets. $Y_2$ vu comme $\Z^2$ (le réseau à coordonnées
      entières) est aussi un revêtement à une infinité de feuillets.
    \end{enumerate}
  \end{exs}



  \begin{lem} \label{lem:lemme-A}
    Soit $\substack{Y\\\downarrow\\ X}p$ un revêtement. Soit $Z$ un espace connexe et soient $f_0, f_1 : Z
    \to Y$ deux applications continues avec $p \circ f_0 = p \circ f_1$. Alors $\{z \in Z | f_0(z) = f_1(z)\}
    = \varnothing$ ou $Z$.
    \begin{center}
      \begin{tikzpicture}
        \draw (0,0) node[scale=0.8]{$\bullet$} node[above]{$Z$};
        \draw (0,-1) node[scale=0.8]{$\bullet$} node[right]{$Y$};
        \draw (0, -2) node[scale=0.8]{$\bullet$} node[right]{$X$};
        \draw[->, >=latex] (0,0) to[bend right] node[midway, left] {$f_0$} (0, -1);
        \draw[->, >=latex] (0,0) to[bend left] node[midway, right] {$f_1$} (0, -1);
        \draw[->, >=latex] (0,-1) to node[midway, left] {$p$} (0, -2);        
      \end{tikzpicture}
    \end{center}
  \end{lem}

  \begin{preuve}
    Exercice.
    \begin{center}
      \begin{tikzpicture}[math3d]
        % Draw helix
        \draw [domain=-2*pi:2*pi, samples=80, smooth] plot ({cos(\x r)}, {sin(\x r)}, \x/pi) ;
        % Draw circle
        \draw (0,0, -4) circle(1);
        % Draw alpha on helix
        \draw [domain=0:pi, samples=80, smooth, color = OliveGreen!80, line width = 2pt] plot ({cos(\x r)},
        {sin(\x r)}, \x/pi) node[xshift = 0.5cm]{$\tilde{\alpha}$};
        % Draw alpha on circle
        \draw [domain=0:pi, samples=80, smooth, color=OliveGreen!80, line width = 2pt] plot ({cos(\x r)},
        {sin(\x r)}, -4);
        % Draw points
        \draw[color=OliveGreen!80] (0, 1, -4) node[right]{$\alpha$};
        \draw[color=OliveGreen!80] (1, 0, -4) node[scale=0.8]{$\bullet$} node[below]{$x_0$};
        \draw[color=OliveGreen!80] (1, 0, 0) node[scale=0.8]{$\bullet$} node[below]{$y_0$};
        % Draw arrow from helix to circle
        \draw[->, >=latex] (0, 0, -2.5) to[bend left] node[midway, left]{$p$} (0,0,-3.5);
        %Second Figure
        \draw (0, 6, 1.3) node{$Z = [0,1]$};
        \draw[->, >=latex] (0, 5.8, 1) to node[midway, left]{$\tilde{\alpha_0}$} (0, 5.8, 0);
        \draw[->, >=latex] (0, 6.2, 1) to node[midway, right]{$\tilde{\alpha_1}$} (0, 6.2, 0);
        \draw (0, 6, -0.3) node{$Y$};
        \draw[->, >=latex] (0, 6, -0.6) to node[midway, left]{$p$} (0, 6, -1.3);
        \draw (0, 6, -1.6) node{$X$};
      \end{tikzpicture}
    \end{center}
  \end{preuve}

  \begin{lem}[relèvement des chemins] \index{Relèvement!de chemins} \label{lem:lemme-B}
    Soient $x_0 \in X$, $y_0 \in p^{-1}(x)$. Pour tout chemin $\alpha: [0,1] \to X$ avec $\alpha(0) = x_0$, il
    existe un unique chemin $\tilde{\alpha} : [0,1] \to Y$ avec $\tilde{\alpha}(0) = y_0$ et $p \circ
    \tilde{\alpha} = \alpha$. On appelle $\tilde{\alpha}$ le \emph{relèvement} de $\alpha$.
  \end{lem}

  \begin{preuve}
    Commençons par montrer l'unicité. Elle résulte du Lemme \ref{lem:lemme-A}. Supposons que $Z = [0,1]$ et
    que $\tilde{\alpha_0}$ et $\tilde{\alpha_1}$ sont deux relèvements de $\alpha$, $\tilde{\alpha_0},
    \tilde{\alpha_1} : Z = [0,1] \to Y$ avec $\tilde{\alpha_0}(0) = \tilde{\alpha_1}(0) = y_0$. Alors par le
    Lemme \ref{lem:lemme-A}, on a que $\tilde{\alpha_0}(z) = \tilde{\alpha_1}(z)$ pour tout $z \in Z$.

    La partie existence est en exercice.
   \end{preuve}

   \begin{lem}[relèvement des homotopies] \index{Relèvement!d'homotopies} \label{lem:lemme-C}
     Soient $\alpha_0, \alpha_1 : [0,1] \to X$ avec $\alpha_0 (0) = \alpha_1(0) = x_0$ et $\alpha_0(1) =
     \alpha_1(1)$. Soient $\tilde{\alpha_0}, \tilde{\alpha_1}$ les relevés par $y_0$. Si $\alpha_0 \sim
     \alpha_1$ dans $X$, alors $\tilde{\alpha_0} \sim \tilde{\alpha_1}$ et ont la même extrémité.
   \end{lem}

   \begin{preuve}
     cf. feuille annexe.
   \end{preuve}

   
   \begin{theo} \label{thm:thm-1}
     Soient $\substack{Y\\\downarrow\\ X}p$ un revêtement, $y_0 \in p^{-1}(x_0)$. Alors 
       \[p_\ast : \Pi_1(Y, y_0) \to \Pi_1(X, x_0)\]
     est injective (si $f: X \to Y$, on définit $f_\ast: \Pi_1(X, x_0) \to \Pi_1(Y, y_0)$ par $[\gamma] \to [f
     \circ \gamma]$). Ainsi $p_\ast(\Pi_1(Y, y_0))$ est un sous-groupe de $\Pi_1(X, x_0)$.
   \end{theo}

   \begin{preuve}
     Soit $[\tilde{\alpha}] \in \Pi_1(Y, y_0)$ avec $p_\ast[\tilde{\alpha}] = [\epsilon_{x_0}]$. Par le Lemme
     \ref{lem:lemme-B}, $\tilde{\alpha}$ est l'unique relèvement de $\alpha = p \circ \tilde{\alpha}$ (et
     $\epsilon_{y_0}$ est l'unique relèvement de $\epsilon_{x_0}$). Par le Lemme \ref{lem:lemme-C}, une
     homotopie entre $\alpha$ et $\epsilon_{x_0}$ se relève en une homotopie entre $\tilde{\alpha}$ et
     $\epsilon_{y_0}$, donc $[\tilde{\alpha}] = [\epsilon_{y_0}]$, ce qui montre que $p_\ast$ est injective.
   \end{preuve}

   Ce théorème nous dit qu'un revêtement de $X$ nous donne un sous-groupe de $\Pi_1(X)$. On a aussi une
   réciproque qui est le théorème suivant.

   \begin{theo} \label{thm:thm-2}
     Soit $X$ connexe par arcs, localement connexe par arcs (graphe). Alors pour $H$ un sous-groupe de
     $\Pi_1(X, x_0)$, il y a un revêtement $\substack{X_H\\\downarrow\\ X}p$ tel que 
       \[p_\ast(\Pi_1(X_H, \tilde{x_0})) \cong H.\]
   \end{theo}
   
   Ceci veut dire que pour un sous-groupe de $\Pi_1(X)$, on peut trouver un revêtement de $X$.

   \begin{rem}
     \begin{enumerate}
     \item $X_H$ est unique à isomorphisme près!
     \item On a donc un dictionnaire entre revêtements et sous-groupes de $\Pi_1(X)$. \qedhere
     \end{enumerate}
   \end{rem}



   \begin{theo}[de Nielsen-Schreier] \index{Théorème!de Nielsen-Schreier}
     Soit $F_n$ le groupe libre avec $n$ générateurs, et soit $H$ un sous-groupe de $F_n$. Alors
     \begin{enumerate}
     \item $H$ est libre;
     \item si $[F_n : H] = k$ (index de $H$ dans $F_n$), alors $H \cong F_{k(n-1)+1}$, c'est-à-dire que $H$
       est libre sur $k(n-1)+1$ générateurs.
     \end{enumerate}
   \end{theo}

   Pour la deuxième partie de la preuve, on a besoin de la proposition suivante.

   \begin{prop} \label{prop:prop-1}
     Le nombre de feuilles $\rev{Y}{X}$ est égal à
       \[\left[ \Pi_1(X, x_0)\ :\ p_\ast\left(\Pi_1(Y, y_0) \right) \right].\]
   \end{prop}

   \begin{preuve}
     Exercice 1, série 6.
   \end{preuve}

   \begin{preuve}[du théorème de \textsc{Nielsen-Schreier}]
     \begin{enumerate}
     \item $F_n$ libre peut être vu comme le groupe fondamental d'un bouquet à $n$ cercles. Pour chaque
       sous-groupe $H$ de $F_n$, on a par le Théorème \ref{thm:thm-2} un revêtement $X_H$ tel que
       $p_\ast(\Pi_1(X_H)) = H$. On a vu que $p_\ast$ est injective, donc on a vraiment l'isomorphisme
       $\Pi_1(X_H) \cong H$.

       Tout revêtement d'un graphe est un graphe, alors $X_H$ est aussi un graphe. Mais le groupe fondamental
       d'un graphe est toujours libre, et ainsi $H$ est libre.


     \item Soit $\F_n$ le groupe fondamental d'un bouquet à $n$ boucles. Pour $H$ un sous-groupe de $\F_n$, il
       y a un revêtement $X_H$ tel que $\Pi_1(X_H) \cong H$. Si $[\F_n : H] = k$, par la proposition 1 on a
       que $X_H$ est un revêtement à $k$ feuillets de $X$.
       Ainsi $X_H$ est un graphe à $k$ sommets et $k\cdot n$ arêtes. Ainsi $H = \Pi_1(X_H) \cong \F_{kn-k+1} =
       \F_{k(n-1)+1}$
       où $kn$ est le nombre d'arête et $k$ est le nombre de sommets.
     \end{enumerate}
   \end{preuve}

   \begin{ex}
     Soit $n=2$. Alors $\F_2$ est le groupe fondamental du bouquet à deux boucles, qu'on appelle $a_1$ et
     $a_2$. Pour $k = 3$, on a le revêtement $X_H$ suivant:
     \begin{center}
       \begin{tikzpicture}
         \node[scale=0.8] (a) at (0, 0) {$\bullet$};
         \node[scale=0.8] (b) at (4, 0) {$\bullet$};
         \node[scale=0.8] (c) at (8, 0) {$\bullet$};
         \draw[->, >=latex, color = OliveGreen] (c.center) to[bend right] node[midway, above]{$a_1$} (a.center);
         \draw[->, >=latex, color = OliveGreen] (c.center) to[bend left] node[midway, below]{$a_2$} (a.center);
         \draw[->, >=latex, color = RoyalBlue] (a.center) to[bend left] node[midway, above]{$a_1$} (b.center);
         \draw[->, >=latex, color = RoyalBlue] (a.center) to[bend right] node[midway, below]{$a_2$} (b.center);
         \draw[->, >=latex, color = red] (b.center) to[bend left] node[midway, above]{$a_1$} (c.center);
         \draw[->, >=latex, color = red] (b.center) to[bend right] node[midway, below]{$a_2$} (c.center);
       \end{tikzpicture}
     \end{center}
     $X_H$ a $k$ sommets de degré $2n$ et a $\frac{k\cdot 2n}{2} = k\cdot n$ arêtes.
   \end{ex}

    
    
    


    
      
  




%%% Local Variables:
%%% mode: latex
%%% TeX-master: "../GAC_cours.tex" 
%%% End:

% Chapter 6: Transformations de Tietze
%---------------------------------------------------------%
%______//------             GAC             ------\\______%
%______||------         Chapitre 6          ------||______%
%______\\------  Transformations de Tietze  ------//______%
%---------------------------------------------------------%

\chapter{Transformations de Tietze}
  
  \begin{defi}
  Soit $\langle X_1 , \ldots, X_n | \underbrace{r_1, \ldots, r_m}_{R}\rangle$ une présentation finie d'un
  groupe $G$. Les transformations suivantes, appelées \emph{transformation de Tietze} \index{Transformation de
    Tietze}, changent la présentation sans changer le groupe.
  \end{defi}



   %Algorithme 
   \begin{algorithm}
     \caption{Première transformation de Tietze}
     \label{alg:trans-tietze-1}
     \begin{algorithmic}
       \State $T_1$ ou $R^+$: Ajouter à la présentation de $G$ un relateur $r_{m+1}$ qui appartient à la
       clotûre normale de $R$ (notée $\overline{R}$, $\lhd R \rhd$ ou $gp_G(R)$).
       \State Soit $r_{m+1} \in \overline{R} \setminus R$: $\langle X | R \rangle \xrightarrow{R^+, T_1}
       \langle X | R \cup \{r_{m+1}\} \rangle$.
     \end{algorithmic}
   \end{algorithm}


   \begin{ex}
     Considérons $\Z^2 = \langle a, b | aba^{-1}b^{-1} \rangle \xrightarrow{R^+} \langle a, b | [a,b], [a,b]^2
     \rangle$.
   \end{ex}


   \begin{algorithm}
     \caption{Deuxième transformation de Tietze}
     \label{alg:trans-tietze-2}
     \begin{algorithmic}
       \State $R^-$: Opération inverse de $R^+$.
       \State Soit $r \in R \setminus \overline{R \setminus \{r\}}$. Alors $\langle X | R \rangle
       \xrightarrow{R^-} \langle X | R \setminus \{r\} \rangle$.
     \end{algorithmic}
   \end{algorithm}


   \begin{algorithm}
     \caption{Troisième transformation de Tietze}
     \label{alg:trans-tietze-3}
     \begin{algorithmic}
       \State $X^+$: Ajouter à la présentation de $G$ un générateur $x_{n+1}$ ainsi qu'une relation $x_{n+1} = w(x_1,
       \ldots, x_n)$ (un mot sur $x_1, \ldots, x_n$).
       \State $\langle X | R \rangle \xrightarrow{X^+} \langle X, x_{n+1} | R \cup \{ x_{n+1}w^{-1}(x_1, \ldots, x_n)\} \rangle$
     \end{algorithmic}
   \end{algorithm}


   \begin{algorithm}
     \caption{Quatrième transformation de Tietze}
     \label{alg:trans-tietze-4}
     \begin{algorithmic}
       \State $X^-$: Opération inverse de $X^+$.
       \State Soit $y \in X$, $w \in \langle X \setminus \{y\} \rangle$ et $y^{-1}w$ est le seul mot dans $R$
       qui contient $y$. Alors
       \State $\langle X | R \rangle \xrightarrow{X^-} \langle X \setminus \{y\} | R \setminus \{y^{-1}w\} \rangle$.
     \end{algorithmic}
   \end{algorithm}

   \begin{ex}
     Soit $G = \langle x,y | xyx = yxy \rangle$. C'est le groupe fondamental du noeud de trèfle. On va
     utiliser les transformations de Tietze. On a
     \begin{align*}
       \langle x,y | xyx = yxy \rangle &\xrightarrow{X^+} \langle x,y,a,b | xyx = yxy, a=xy, b=xyx \rangle\\
       &\xrightarrow{R^+} \langle x,y,a,b | xyx=yxy, a=xy, b=xyx, x=a^{-1}b, y = b^{-1} b^{-1}a^2, a^3 = b^2
         \rangle & a^3 = xyxyxy\\
       &\xrightarrow{R^-} \langle x,y,a,b | a^3 = b^2, x = a^{-1}b, y = b^{-1}a^2 \rangle\\
       &\xrightarrow{X^-} \langle a,b | a^3 = b^2 \rangle.
     \end{align*}
     Cette dernière présentation correspond au produit libre amalgamé.
   \end{ex}


   \begin{prop}[de \textsc{Tietze}] \label{prop:de-Tietze} \index{Proposition!de \textsc{Tietze}}
     Les transformations de Tietze ne changent pas le groupe.
   \end{prop}

   \begin{preuve}[pour $X^+$]
     Supposons que $G = \langle X | R \rangle$, $y$ est un symbole qui n'est pas dans $X$, et $w(X)$ un mot
     réduit de $\F(X)$. On veut montrer que $\langle X, y | R \cup \{y^{-1}w(X)\} \rangle \cong \langle X, R
     \rangle = G$. 

     Soit $\phi : \F(X) \to G$ l'homomorphisme donné par la propriété universelle des groupes libres. Le
     groupe libre $\F(X, y)$ sur $X \cup \{y\}$ est engendré librement par $X \cup
     \{y^{-1}w(X)\}$. C'est-à-dire que $\F(X \cup \{y\}) = \F(X \cup \{y^{-1}w(X)\})$. C'est vrai car à
     partir de $y^{-1}w(X)$, on peut obtenir $y$ (cette inclusion est sensée être facile), et à partir de
     $y$ on peut obtenir $y^{-1}w(X)$. Ainsi on a
       \[X \cup \{y^{-1}w(x)\} \hookrightarrow \F(x,y) = \F(X \cup \{y^{-1}w(X)\}.\]
     Il y a un unique homomorphisme $\phi^1: \F(x,y) \to G$ tel que $\phi^1(x) = \phi(x)$ et
     $\phi^1(y^{-1}w(X)) = 1$ pour $x \in X$.

     \begin{center}
       \begin{tikzcd}
         X \cup \{y^{-1} \omega(X)\} \arrow[r] \arrow[d, "f"] & \F(X,y) = \F(X \cup \{y^{-1}\omega(X)\})
         \arrow[d, "\chi"]\\
         G \arrow[ru, leftarrow, "\phi^1!"] \arrow[r, leftarrow, "\phi"] & \F(X)
       \end{tikzcd}
     \end{center}

     ($f(x) = x$ si $x \in X$ et $1$ si $x = y^{-1} \omega(X)$). L'homomorphisme $\phi^1: \F(X,y) \to G$ se factorise comme $F(X,y) \xrightarrow{\chi} \F(X)
     \xrightarrow{\phi} G$ où $\chi(x) = x$ pour tout $x \in X$ et $\chi(y) = w(x)$. Alors $\phi^1$ est
     surjective et
       \[\ker \phi^1 = \chi^{-1}(\phi^{-1}(1)) = \chi^{-1} \left( gp_{\F(X)}R \right) = gp_{\F(X,y)}(R \cup
       \{y^{-1}w(X)\}).\]
     Ainsi par le premier théorème d'isomorphisme, on a que 
       \[G \cong \F(x,y)/\ker \phi^1 = \langle X,y | R \cup \{y^{-1}w(X)\} \rangle.\]
   \end{preuve}



   \begin{theo}[de \textsc{Tietze}] \label{thm:de-Tietze} \index{Théorème!de \textsc{Tietze}}
     Soient ${\cal P}_1 = \langle X | R \rangle$ et $ {\cal P}_2 = \langle Y | S \rangle$ des présentations
     finies pour un groupe $G$. Alors il existe une suite finie de transformations de Tietze qui transforment
     ${\cal P}_1$ en ${\cal P}_2$.
   \end{theo}

   \begin{preuve}
     cf. feuille annexe. $G$ est donné par ${\cal P}_1$ et ${\cal P}_2$. Alors chaque $x \in X$ peut être
     écrit comme un mot sur $Y$, et on note $x(Y)$. Alors $X(Y)$ représente tous les mots sur $Y$ qui
     décrivent les éléments de $X$. De la même manière, on définit $y(X)$ et $Y(X)$. 

     On commence avec ${\cal P}_1$ et on utilise les transformations suivantes (voir feuille annexe).

     Intuitivement, on ajoute tous les générateurs $Y$ et on enlève tous les générateurs $X$. On utilise les
     transformations $R^+$ $|X|+|R|+|Y|+|S|$ fois et $R^- 2(|R|+|Y|)$ fois. Donc il y a un nombre fini de transformations de ${\cal
       P}_1$ à ${\cal P}_2$.
   \end{preuve}

   \begin{cor}
     On peut énumérer toutes les présentations finies d'un groupe $G$ à partir d'une présentation quelconque
     pour $G$.
   \end{cor}

   \begin{prop}
     Si le groupe $G$ a une présentation finie $\langle X_1 |R \rangle$ et une présentation infinie $\langle
     X_2 | S \rangle$ où $S$ est infini, alors il existe un entier $n$ tel que $\langle X_2 | s_1, \ldots, s_n
     \rangle$ est une présentation finie pour $G$.
   \end{prop}

   \section{Algorithme de Todd-Coxeter (1936) (Coset enumeration)}
   \label{sec:algorithme-de-todd-loxeter}

   Étant donné un groupe $G$, défini par une présentation finie $G = \langle X, R \rangle$, et un sous-groupe
   $H$ de $G$ d'indice fini dans $G$, on souhaite énumérer les éléments du quotient $G/H$ et décrire l'action
   de $G$ sur $G/H$. 

   \subsection{Version basique}
   \label{sec:version-basique-todd-coxeter}

      Si $H = \{1\}$, l'algorithme va énumérer les éléments de $G$, si $G$ est fini.

     \begin{algorithm}
       \caption{Algorithme de Todd-Coxeter (basique)}
       \label{alg:todd-coxeter-basique}
       \begin{algorithmic}
         \State $\forall r \in R$, créer un tableau de $|r|+1 colonnes$.
         \State Si $r = x_1 \cdots x_n$, le tableau est
         \State
         \begin{tabular}{c p{2cm} c}
           \begin{tabular}{|ccccccccccc|}
             \hline
             & $x_1$ & & $x_2$ & & $\cdots$ & & $x_1^{-1}$ & & $x_n$ & \\
             \hline
             1 & - & 2 &  & & $\cdots$ & 2 & & 1 & & 1 \\
             2 &  & & & & & & & & & 2\\
             \hline
           \end{tabular}
             & &
                 \begin{tabular}{|c|c|}
                   \hline
                   Définition & Bonus \\
                   \hline
                   $1x_1 = 2$ & \\
                   \hline
                 \end{tabular}
         \end{tabular}
         \State On pose $1$ dans la première et la dernière colonne (1 pour $1_G$)
         \State On pose $2$ à la droite de $1$, ça s'appelle la \og définition\fg de $2$ et on le pose dans un
         autre tableau, qui s'appelle le \emph{tableau de définitions}. \index{Tableau de définitions} La
         notation $1 x_1 = 2$ ou $2 x_1^{-1} = 1$ ($1 = 1_G$, $2 = x_1$).
         \State On pose $2$ dans la première et la dernière colonne, deuxième ligne.
         \State S'il y a un 2 à la gauche de $x_1^{-1}$, on pose $1$ à droite de ce 2.
         \State S'il y a un 2 à la droite de $x_1$, on pose $1$ à la gauche de $x_1$.
         \State On pose $3, 4, \ldots$ dans le tableau jusqu'à ce qu'il n'y ait plus d'espaces vides.
       \end{algorithmic} 
     \end{algorithm}
     

     \begin{rem} \label{rem:rem-1}
       Chaque nombre $1, 2, 3, \ldots$ représente un élément de $G$.
     \end{rem}
     

     \begin{rem} \label{rem:rem-2}
       Supposons qu'on ait une définition $i x_l = j$, et dans le tableau on ait aussi $k$ à la droite de
       $x_{l+1}$, on a $kx_{l+1}^{-1} = j \iff jx_{l+1} = k$. On appelle cela un \emph{bonus}. \index{Bonus}
       \begin{center}
         \begin{tabular}{|ccccc|}
           \hline
           & $x_p$ & & $x_{p+1}$ & \\
           \hline
           i & $-$ & j & $=$ & k\\
           \hline
         \end{tabular}
       \end{center}
       Dans le tableau, on note $-$ quand on a une définition, et $=$ lorsqu'on a un bonus.
     \end{rem}


     \begin{ex}
       Soit $G = \langle x\ |\ x^4 = 1\rangle \cong \Z/4\Z$. L'unique relateur est $x^4$.
       \begin{center}
         \begin{tabular}{c p{2cm} c}
           \begin{tabular}{|ccccccccc|}
             \hline
             & $x$ & & $x$ & & $x$ & & $x$ &  \\
             \hline
             1 & $-$ & 2 & $-$ & 3 & $-$ & 4 & $=$ & 1\\
             2 & & 3 & & 4 & & 1 & & 2\\
             3 & & 4 & & 1 & & 2 & & 3\\
             4 & & 1 & & 2 & & 3 & & 4\\
             \hline
           \end{tabular}
           & & 
               \begin{tabular}{|c|c|}
                 \hline
                 Définition & Bonus\\
                 \hline
                 $1x = 2$ & \\
                 $2x = 3$ & \\
                 $3x = 4$ & $4x = 1$\\
                 \hline
               \end{tabular}
         \end{tabular}
       \end{center}
       Donc $|G| = 4$, avec $1 = 1_g$, $2 = x$, $3 = x^2$ et $4 = x^3$.
     \end{ex}

     


     \begin{theo}
       Si $G$ est fini, l'algorithme de Todd-Coxeter s'arrête avec un tableau complet pour chaque relateur
       après un nombre fini d'étapes.

       L'ensemble de sortie donné par l'algorithme contient tous les éléments de $G$, mais aussi l'action à
       droite de générateurs de $G$ sur $G$.
     \end{theo}


     \subsection{Version générale}

       Soit $G = \langle X | R \rangle$, $H$ un sous-groupe de $G$ et $H \neq \{1\}$. Si $H = \langle Y
       \rangle$, alors $H$ est donné par un ensemble $Y$ de générateurs qui sont des mots sur $X$.

       \paragraph{But:} On obtient $|G:H|$ si $|G:H| < \infty$, l'action de $G$ sur $G/H$, et un ensemble de
       représentants de classes à droite de $H$.

       
       \begin{algorithm}
       \caption{Algorithme de Todd-Coxeter}
       \label{alg:todd-coxeter}
       \begin{algorithmic}
         \State L'algorithme est le même que celui de la version basique, mais on ajoute un tableau pour chaque générateur de $H$.
         \State L'algorithme se termine quand tous les espaces dans les tableaux des relateurs sont remplis.
         \State $|G:H| = $ nombre de lignes en chaque tableau.
       \end{algorithmic} 
     \end{algorithm}

     \begin{ex}
       Soit $G = \langle x | x^6 = 1 \rangle$, $H = \langle x^3 \rangle$. Ici, $1 = H$.
       \begin{center}
         \begin{tabular}{c p{2cm} c}
           \begin{tabular}{|ccccccccccccc|}
             \hline
             & $x$ & & $x$ & & $x$ & & $x$ & & $x$ & & $x$ & \\
             \hline
             1 & $-$ & 2 & $-$ & 3 & & 1 & & 2 & & 3 & & 1\\
             \hline
             2 & & 3 & & 1 & & 2 & & 3 & & 1 & & 2\\
             \hline
             3 & & 1 & & 2 & & 3 & & 1 & & 2 & & 3 \\
             \hline
           \end{tabular}
               & &
                   \begin{tabular}{|ccccccc|}
                     \hline
                     & $x$ & & $x$ & & $x$ & \\
                     \hline
                     1 & & 2 & & 3 & $=$ & 1\\
                     \hline
                   \end{tabular}
         \end{tabular}
       \end{center}
       
       \begin{center}
         \begin{tabular}{|c|c|}
           \hline
           Définition & Bonus \\
           \hline
           $1x = 2$ & \\

           $2x = 3$ &  $3x = 1$\\
           \hline
         \end{tabular}
       \end{center}

       
       On a 3 lignes donc $|G:H| = 3$. Les classes à droites sont $1 = H$, $2 = Hx$ et $3 = Hx^2$. Les
       représentants sont $\{1, x, x^2\}$.
     \end{ex}


     \begin{ex}
       Soit $G = \langle x, y | x^3 = 1, y^3 = 1, (xy)^2 = 1 \rangle$ et $H = \langle x \rangle$.

       Tableaux pour $H$ et $x^3 = 1$:
       
       \begin{center}
         \begin{tabular}{c p{2cm} c}
           \begin{tabular}{|ccc|}
             \hline
             & $x$ & \\
             \hline
             1 & $=$ & 1 \\
             \hline
           \end{tabular}

           & &
               \begin{tabular}{|ccccccc|}
                 \hline
                 & $x$ & & $x$ & & $x$ & \\
                 \hline
                 1 & & 1 & & 1 & & 1\\
                 2 & & 3 & $-$ & 4 & $=$ & 2\\
                 3 & & 4 & & 2 & & 3\\
                 4 & & 2 & & 3 & & 4\\
                 \hline
               \end{tabular}
         \end{tabular}
       \end{center}

       Tableaux pour $y^3$ et $(xy)^2$

       \begin{center}
         \begin{tabular}{c p{2cm} c}
           \begin{tabular}{|ccccccc|}
             \hline
             & $y$ & & $y$ & & $y$ & \\
             \hline
             1 & $-$ & 2 & $-$ & 3 & $=$ & 1\\
             2 & & 3 & & 1 & & 2\\
             3 & & 1 & & 2 & & 3\\
             4 & & 4 & & 4 & & 4\\
             \hline
           \end{tabular}
               & &
                   \begin{tabular}{|ccccccccc|}
                     \hline
                     & $x$ & & $y$ & & $x$ & & $y$ & \\
                     \hline
                     1 & & 1 & & 2 & & 3 & & 1\\
                     2 & $=$ & 3 & & 1 & & 1 & & 2\\
                     3 & & 4 & & 4 & & 2 & & 3\\
                     4 & & 2 & & 3 & & 4 & & 4\\
                     \hline
                   \end{tabular}
         \end{tabular}
       \end{center}
       
      
       Tableau des définitions et bonus:
       \begin{center}
         \begin{tabular}{|c|c|}
           \hline
           Définition & Bonus\\
           \hline
               & $1x = 1$\\
           $1y = 2$ & \\
           $2y = 3$ & $3y = 1$, $2x = 3$\\
           $3x = 4$ & $4x = 2$, $4y = 4$\\
           \hline
         \end{tabular}
       \end{center}

       On voit donc que $|G:H| = 4$ et les classes de $H$ sont $1 = H$, $2 = Hy$, $3 = Hy^2$ et $4 =
       Hy^2x$. Il y a une action de $G$ sur $G/H = \{1,2,3,4\}$, c'est-à-dire qu'il y a un homomorphisme
       $\alpha : G \to Sym(4),\ x \mapsto
       \begin{pmatrix}
         1 & 2 & 3 & 4\\
         1 & 3 & 4 & 2
       \end{pmatrix} = \alpha(x)$ et $y \mapsto
       \begin{pmatrix}
         1 & 2 & 3 & 4 \\
         2 & 3 & 1 & 4
       \end{pmatrix} = \alpha(y)$. Ainsi l'ordre de $x$ $ord(x) \geq ord(\alpha(x)) = 3$, mais $x^3 = 1$ dans
       $G$ donc $ord(x) = 3$. Comme $|H| = |\langle x \rangle| = 3 \Rightarrow |G| = |H||G:H| = 3 \cdot 4 =
       12$. Mais $\langle (234), (123)\rangle \cong Alt(4)$ et $\alpha$ est injective, surjective et ainsi $G
       \cong Alt(4)$.
     \end{ex}


     \begin{ex}
       Soit $G = F(2,5) = \langle x, a, b, c, d | xa = b, ab = c, bc = d, cd = x, dx = a \rangle$ et soit $H =
       \langle x \rangle$. Le tableau pour $H$ est simplement $1x = 1$, et donc c'est notre premier bonus.
       
       Tableaux pour $xa = b$ et $ab = c$.
       \begin{center}
         \begin{tabular}{c p{2cm} c}
           \begin{tabular}{|ccccccc|}
             \hline
               & $x$ &   & $a$ &   & $b^{-1}$ & \\
             \hline
             1 &     & 1 & $-$ & 3 & $=$     & 1\\
             2 &     & 3 &     &   &         & 2\\
             3 &     &   &     & 2 &         & 3\\
             \hline
           \end{tabular}
               & &
                   \begin{tabular}{|ccccccc|}
                     \hline
                       & $a$ &   & $b$ &   & $c^{-1}$ & \\
                     \hline
                     1 &     & 3 & $=$ & 2 &         & 1\\
                     2 &     & 1 &     & 3 & $=$     & 2\\
                     3 &     &   &     &   &         & 3\\
                     \hline
                   \end{tabular}
         \end{tabular}
       \end{center}

       Tableaux pour $bc = d$ et $cd = x$.
       \begin{center}
         \begin{tabular}{c p{2cm} c}
           \begin{tabular}{|ccccccc|}
             \hline
               & $b$ &   & $c$ &   & $d^{-1}$ & \\
             \hline
             1 &     & 3 & $\equiv$ & 2 &    & 1\\
             2 &     &   &     & 1 &         & 2\\
             3 &     & 2 &     & 3 &   $=$   & 3\\
             \hline
           \end{tabular}
               & &
                   \begin{tabular}{|ccccccc|}
                     \hline
                       & $c$ &   & $d$ &   & $x^{-1}$ & \\
                     \hline
                     1 & $-$ & 2 & $=$ & 1 &         & 1\\
                     2 &     & 3 &     & 3 & $=$     & 2\\
                     3 &     &   &     &   &         & 3\\
                     \hline
                   \end{tabular}
         \end{tabular}
       \end{center}

       Tableau pour $dx = a$:
       \begin{center}
         \begin{tabular}{|ccccccc|}
           \hline
             & $d$ &   & $x$ &   & $a^{-1}$ & \\
           \hline
           1 & $=$ & 2 &     & 3 &         & 1\\
           2 &     & 3 &     & 1 &   $=$   & 2\\
           3 &     & 3 &     &   &         & 3\\
           \hline
         \end{tabular}
       \end{center}
       
       Tableau des définitions et bonus et tableau des relations
       \begin{center}
         \begin{tabular}{c p{2cm} c}
           \begin{tabular}{c|c}
             \hline
             Définition & Bonus\\
             \hline
                        & $1x = 1$\\
             \underline{$1c = 2$}   & $2d = 1$, $2a = 1$\\
             $1a = 3$   & $1b = 3$, $3b = 2$, $2c = 3$, $3d = 3$\\
                        & $2x = 3$, $1d = 2$, \underline{$3c = 2$}\\
             \hline
           \end{tabular}
           & & 
               \begin{tabular}{c|c|c|c|c|c}
                   & $x$ & $a$ & $b$ & $c$ & $d$\\
                 \hline
                 1 &  1  &  3  &  3  &  2  &  2 \\
                 2 &  3  &  1  &     &  3  &  1 \\
                 3 &     &     &  2  &     &  3
               \end{tabular}
         \end{tabular}
       \end{center}

       À ce point, on déduit que $1 = 2c^{-1} = 3$, d'où $3 = 1b = 3b = 2$ et les tableaux se réduisents
       chacun à une ligne. La second tableau de référence nous dis que chacun des cinq générateurs fixe 1
       [voir feuille annexe pour détail, pas trop compris pourquoi], ainsi $F(2, 5) = \langle x \rangle$ et
       est donc abélien. Comme on sait déjà que le \og derived factor group\fg de $F(2,5)$ est $Z_{11}$, on en
       déduit que $F(2,5) \cong Z_{11}$.
     \end{ex}
     




     

%%% Local Variables:
%%% mode: latex
%%% TeX-master: "../GAC_cours.tex" 
%%% End:

% Chapter 7: Graphes de Cayley
%---------------------------------------------------------%
%______//------             GAC             ------\\______%
%______||------         Chapitre 7          ------||______%
%______\\------      Graphes de Cayley      ------//______%
%---------------------------------------------------------%

\chapter{Graphes de Cayley (groupes comme espaces métriques)}
\label{sec:graphes-de-Cayley}

  Soit $G$ un groupe, $S \subset G$ une partie symétrique ($s \in S \Rightarrow s^{-1} \in S$) et $1 \notin
  S$.

  \begin{defi} \index{Graphe!de Cayley}
    Le \emph{graphe de Cayley}, noté $\Gamma(G, S)$ est le graphe dont l'ensemble des sommets est $G$ est
    l'ensemble des arêtes est $E = \{(x,y) | xy^{-1} \in S \iff \exists s \in S\ :\ y = xs\}$.

    Deux sommets sont \emph{voisins}, et on les notes $x \sim y$, si $y$ s'obtient à partir de $x$ par
    multiplication par un élément de $S$.
  \end{defi}

  \begin{exs}
    \begin{enumerate}
    \item Soit $G = \Z/6\Z$ et $S = \{x, x^{-1}\}$, pour $x = 1$ et $x^{-1} = 5$.

    \item Soit $G = \Z/6\Z$ et $S = \{2, -2\}$.

    \item Soit $G = \Z/6\Z$ et $S = \{3 = -3\}$.

    \item Soit $G = \Z/6\Z$ et $S = \{2, -2, 3 = -3\}$.

    \item Soit $G = \Z$ et $S = \{1, -1\}$.

    \item Soit $G = \Z^2$ et $S = \{(\pm 1, 0), (0, \pm 1)\}$.

    \item Soit $G = \F_2 = \F(a,b)$ le groupe libre avec 2 générateurs et soit $S = \{a^{\pm 1}, b^{\pm 1}\}$.
    \end{enumerate}
  \end{exs}

  \begin{propri}
    \begin{enumerate}
    \item $\Gamma(G,S)$ est $k$-régulier, où $k = |S|$ (c'est-à-dire tout sommet a $k$ voisins).
    \item $\Gamma(G,S)$ et connexe $\iff$ $S$ engendre $G$.
    \end{enumerate}
  \end{propri}

  \begin{defi} \index{Arbre}
    Un \emph{arbre} un graphe connexe sans chemins fermés.
  \end{defi}

  \begin{prop}
    Soit $G$ un groupe et $S \subseteq G$ un ensemble. Alors $G cong \F(S)$ (le groupe libre sur $S$) si et
    seulement si $\Gamma(G, S)$ est un arbre.
  \end{prop}


  \begin{defi}
    Soient $\Gamma_1 = (V_1, E_1)$ et $\Gamma_2 = (V_2, E_2)$ deux graphes. Un \emph{morphisme de graphe}
    \index{Morphisme!de graphe} est une application $\phi: \Gamma_1 \to \Gamma_2$, $\phi \big|_V : V_1 \to
    V_2$, $\phi \big|_E : E_1 \to E_2$ telle que $(\phi(v), \phi(v')) \in E_2 \iff (v, v') \in E_1$.

    Si $\phi$ est bijective et $\Gamma_1 = \Gamma_2$, $\phi$ s'appelle un \emph{automorphisme de graphe}.
    \index{Automorphisme!de graphe}

    L'ensemble de tous les automorphisme de $\Gamma$, noté $Aut(\Gamma)$ forme un groupe.
  \end{defi}


  \begin{theo}
    Soit $G$ un groupe dénombrable. Alors il y a un graphe connexe $X$ tel que $G \cong Aut(X)$.
  \end{theo}

  \begin{preuve}
    Soit $S = \{s_1, s_2, \ldots \}$ un ensemble dénombrable de générateurs, c'est-à-dire que $G = \langle S
    \rangle$. Soit $X_0 = \Gamma(G,S)$ le graphe de Cayley de $G$ par rapport à $S$. Il faut se convaincre que
    $G \not\cong Aut(X_0)$.

    Pour chaque $i \geq 1$, soit $T_i$ un arbre fini (qui correspond à $s_i$)
    \begin{center}
      Figure de $T_i$ ici.
    \end{center}
    Si $i \neq j$, on a que $T_i \not\cong T_j$, car on n'a pas le même nombre de sommets dans $T_i$ et
    $T_j$. On doit montrer que $Aut(T_i) = \{id\}$ (exercice). 

    Dans le graphe de Cayley $X_0$, on remplace chaque arête $s_i$ par $T_i$, par exemple
    \begin{center}
      Dessiner exemple ici
    \end{center}
    et on obtient un graphe $X$. On a ainsi une expansion de $X_0$ vers $X$ ainsi qu'une contraction de $X$
    vers $X_0$ (en remplaçant $T_i$ par $s_i$). On observe que chaque automorphisme $\phi: X \to X$ induit un
    automorphisme $\phi_0: X_0 \to X_0$. 
    \begin{enumerate}
    \item Si $\gamma \in Aut(X)$ fixe $x \in V(X)$ ($\gamma(x) = x$), alors $\gamma = id_X$. En effet, si $x
      \in V(X) \setminus V(X_0)$ et $\gamma(x) = x$, alors $x \in V(T_i) \setminus \{a_i, b_i\}$. Ainsi
      $\gamma(T_i) = T_i$. Supposons que $x \in V(X_0)$ et $\gamma(T_{x,j}) = T_{x,j}$ pour chaque $x \in
      T_{x,j}$. L'idée est que si on fixe un tel $x$, on est obligé de fixer l'arbre $T_{x,j}$ (car deux
      arbres différents ne sont pas isomorphes), ainsi on fixe tous les arbres, donc toutes les arêtes et
      ainsi on fixe $xs_i$ pour chaque $i$, et par le Lemme de Zorn (car c'est un arbre infini, donc on doit
      faire ce processus à l'infini), on fixe l'arbre. C'est-à-dire que $\gamma(X) = X$, donc $\gamma = id$.

    \item $Aut(X) \cong G$. Soit $\phi: G \to Aut(X)$, $g \mapsto \phi_g$ où $\phi_g : X \to X$ est une
      extension de $\phi_g^0: X_0 \to X_0$ et $\phi_g^0(v) = gv$ si $v \in V(X_0) = G$. Commençons par montrer
      que $\phi$ est injective. On a
      \begin{align*}
        \phi(g) = \phi(g') &\iff \phi_g(v) = \phi_{g'}(v)\\
        &\iff gv = g'v\\
        &\iff g = g'
      \end{align*}
      Montrons à présent que $\phi$ est surjective. Soit $\psi \in Aut(X)$ avec $\psi(v) = v'$ pour $v,v' \in
      X_0$, alors il existe $g \in G$ tel que $\phi_g(v) = v'$ (par exemple on prend $g = v'v^{-1}$). On a
      $\psi(v) = \phi_g(v)$, et ainsi $(\psi^{-1}\phi_g)(v) = v$ et par la première observation,
      $\psi^{-1}\phi_g = id$ et donc $\psi = \phi_g$. \qedhere
    \end{enumerate}
  \end{preuve}

  \begin{rem}
    Si on change l'ensemble $S$, les graphes de Cayley pour $G$ sont différents entre eux (c'est-à-dire qu'ils
    ne sont pas isomorphes, en général). Mais ils sont \emph{quasi-isométriques} \index{Graphe!quasi-isométrique}
  \end{rem}


  À présent, on supposera toujours que $S$ engendre $G$ (sinon le graphe n'est pas connexe, et on n'a pas des
  bonnes propriétés).

  \begin{defi} \index{Longueur d'un mot $g$}
    Soit $g \in G$. La \emph{longueur} des mots de $g$ est la distance de $g$ à $\epsilon$ dans $\Gamma(G,S)$.
      \[|g|_S := \min\{n \in \N\ |\ g = s_1\cdots s_n,\ s_i \in S\}.\]

    Pour $g \in G$, la \emph{distance} de $x$ a $y$ \index{distance entre deux mots} est celle dans
    $\Gamma(G,S)$, notée $d_S(x,y) = |x^{-1}y|_S$.
  \end{defi}

  \begin{obss}
    \begin{enumerate}
    \item $d_S: G \times G \to \N$ est une distance sur $G$, invariante par l'action à gauche de $G$, $d_S(gx,
      gy) = d_S(x,y)$.

    \item Cette distance dépend du choix d'un système de générateurs $S$. Mais on va voir qu'en regardant le
      groupe \og de loin\fg, plusieurs propriétés importantes ne dépendent pas de $S$ (Gromov, $\sim 1980$).
    \end{enumerate}
  \end{obss}

  \begin{ex}
    Considérons $G = \Z$ et $S = \{\pm 1\}$. Le graphe de Cayley est simplement une droite infinie à gauche et
    à droite. Si à présent on considère $S' = \{\pm 2, \pm 3\}$, le graphe de Cayley devient 
    \begin{center}
    
            \begin{tikzpicture}
              \foreach \k in {-3,-2,...,5} { \draw (\k, 0) node[scale=0.8]{$\bullet$} node[below]{$\k$}; }
              \foreach \k in {-3, -2, ..., 3}{ \draw[color = red] (\k, 0) to[bend left] ({\k+2}, 0); }
              \foreach \k in {-3, -2, ..., 4}{ \draw[color = OliveGreen] (\k, 0) to[bend right] ({\k+3}, 0); }
            \end{tikzpicture}

          \end{center}
  \end{ex}

 
  

  

  




%%% Local Variables:
%%% mode: latex
%%% TeX-master: "../GAC_cours.tex" 
%%% End:

% Chapter 8: Propriétés géométriques
%---------------------------------------------------------%
%______//------             GAC             ------\\______%
%______||------         Chapitre 8          ------||______%
%______\\------   Propriétés géométriques   ------//______%
%---------------------------------------------------------%

\chapter{Propriétés géométriques}
\label{cha:propr-geom}

  Une propriété géométrique est une propriété invariante par quasi-isométries.


  \begin{defi}\index{Groupes!commensurables} 
    Deux groupes sont \emph{commensurables} s'ils possèdent des sous-groupes d'indice fini isomorphes.
  \end{defi}

  \begin{rems}
    \begin{enumerate}
    \item Pour les groupes finiment engendrés, deux groupes commensurables sont quasi-isométriques (à cause du
      corollaire précédent).
    \item En revanche, deux groupes quasi-isométriques n'implique pas qu'ils sont commensurables (en général).
    \end{enumerate}
  \end{rems}

  \begin{exs}
    \begin{enumerate}
    \item Si $F$ est un groupe fini, alors $\Z \times F$ et $D_\infty$ sont commensurables, car ils possèdent
      tous les deux le sous-groupe $\Z$ qui est d'indice fini.

    \item Pour $k, l \geq 2$, $\F_k$ est commensurable à $\F_l$.
    \end{enumerate}
  \end{exs}

  \begin{defi}
    Soit $(P)$ une propriété des groupes finiment engendrés. On dit que $G$ est \emph{virtuellement $(P)$}
    \index{Groupe!virtuellement $(P)$} si $G$ possède un sous-groupe $H$ d'indice fini qui a la propriété $(P)$.
  \end{defi}

  \begin{ex}[virtuellement libre]
    Si $G$ est virtuellement libre, alors $G$ possède un sous-groupe $H$ d'indice fini qui est libre.
  \end{ex}

  \begin{exs}
    \begin{enumerate}
    \item \og Être fini \og est une propriété géométrique. Car $G$ est fini ss'il est qi à $\{1\}$ et par
      transitivité, si $H$ est qi à $G$, il est aussi qi à $\{1\}$ et donc fini.

    \item \og Être cyclique infini\fg{}, c'est-à-dire \og être $\Z$\fg{}, n'est pas une propriété
      géométrique. $\Z$ et $\Z \times \Z/2\Z$ sont qi, mais le second n'est pas cyclique infini. De même pour
      $\Z$ et $D_\infty$.
    \end{enumerate}
  \end{exs}


  \begin{prop}
    \begin{enumerate}
    \item Être virtuellement $\Z$ est une propriété géométrique.
    \item Être virtuellement libre est une propriété géométrique.
    \item Être virtuellement abélien est une propriété géométrique.
    \item Avoir une présentation finie est une propriété géométrique.
    \item Avoir un problème des mots résolubles est une propriété géométrique.
    \item Être virtuellement nilpotent est une propriété géométrique.
    \end{enumerate}
  \end{prop}

  La proposition est difficile à prouver, mais pour quelques assertions, on peut le montrer en utilisant la
  croissance des groupes.

  \section{Croissance des groupes}
  \label{sec:croiss-des-groupes}
  
    Soit $G$ un groupe finiment engendré et $S = S^{-1}$ une partie finie symétrique génératrice de $G$.

    \begin{defi}
      La \emph{fonction de croissance} \index{Fonction!de croissance} de $G$ par rapport à $S$ est
        \[V_S: \N \to \R^+\ (\N^+),\ n \mapsto |B_S(n)| \]
      où $B_S(n) = \{g \in G\ |\ |g|_S \leq n\}$.
    \end{defi}

    \begin{exs}
      \begin{enumerate}
      \item Si $G$ est fini, alors $V_S(n)$ est constant pour $n \gg 0$

      \item Si $G = \Z$ et $S = \{\pm 1\}$,
        \begin{center}
          Dessin ici
        \end{center}
        alors $V_s(n) = 2n + 1$.

      \item Si $G = \Z^2$ avec $S = \{(\pm 1, 0), (0, \pm 1)\}$,
        \begin{center}
          Dessin ici
        \end{center}
        alors $V_S(n) = 1 + 4 \sum_{j=1}^n(n+1-j) = 2n^2 + 2n + 1 \leq (2n+1)^2$ (à vérifier).

      \item Si $G = \Z^d$ avec $S = \{(\pm 1, 0, \ldots, 0), \ldots, (0, 0, \ldots, \pm 1)\}$. Alors
        \[|B_S(n)| \leq (2n+1)^d,\ |B_S(n)| \geq \text{volume de la boule euclidienne de rayon }
        \frac{n}{\sqrt{d}} \cong C_dn^d.\]

      \item Si $G = \F_k$, $S = \{a_1^{\pm 1}, \ldots, a_k^{\pm 1}\}$. Soit $S(n) = \{g \in \F_k\ |\ |g|_S =
        n\}$. Alors $|S(n)| = 2k(2k-1)^{n-1}$ (car on a $2k$ choix pour la première lettre du mot, et ensuite
        comme on ne considère que des mots réduits, on a $2k-1$ choix pour le reste des lettres du mot). Ainsi
          \[V_S(n) = \sum_{i=0}^n |S(i)| = \cdots = \frac{k(2k-1)^n-1}{k-1}.\]
       \item Le groupe de Heisenberg discret.\index{Groupe!de Heisenberg discret}
         Si $A$ est un anneau commutatif à unité, le groupe de Heisenberg sur $A$ est 
         \[Heis(A) = \left\{
           \begin{pmatrix}
             1 & x & z \\ 0 & 1 & y \\ 0 & 0 & 1
           \end{pmatrix}\ |\ x,y,z \in A \right\} \subseteq GL_3(A).
         \]
         Considérons $Heis(\Z) = \langle a,b,c \rangle$.
           \[a =
           \begin{pmatrix}
             1 & 1 & 0 \\ 0 & 1 & 0 \\ 0 & 0 & 1
           \end{pmatrix},\ 
           b =
           \begin{pmatrix}
             1 & 0 & 0 \\ 0 & 1 & 1 \\ 0 & 0 & 1
           \end{pmatrix},\ 
           c =
           \begin{pmatrix}
             1 & 0 & 1 \\ 0 & 1 & 0 \\ 0 & 0 & 1
           \end{pmatrix}.
           \]
         On a que $[a,b] = c$, $[a,c] = [b, c] = 1$ (commutateurs). Ceci nous dit que $c$ commute avec $a$ et $b$.
         
         \textbf{Exercice:} Pour tous $m, n \in \Z$, $a^mb^n = c^{mn}b^na^n$ $(\ast)$ . Indication: démontrer que $a^m b
         = c^m ba^m$.
      \end{enumerate}
    \end{exs}

    \begin{rem}
      $(\ast)$ nous donne que tout $g \in G$ a une expression $g = a^mb^nc^p$ pour $m, n, p \in \Z$.
    \end{rem}

    \begin{prop}[Croissance du groupe de Heisenberg]
      Pour le groupe de Heisenberg, il existe des polynômes $P_1, P_2$ de degré 4 tels que
        \[\forall n \in \N, P_1(n) \leq V_S(n) \leq P_2(n).\]
    \end{prop}

    \begin{preuve}
      \begin{enumerate}
      \item Montrons que si si $g = a^m b^n c^p$, et $|g|_S \leq N$, alors $|m|, |n| \leq N$, $|p| \leq
        N^2$. En effet on au plus $(2N+1)$ choix pour $|m|$ et $|n|$ et $2N^2 + 1$ choix pour $|p|$, donc
          \[V_S(n) \leq (2N+1)^2(2N^2+1).\]
        L'application $\alpha : Heis(\Z) \to \Z^2$, $a^mb^nc^p \mapsto (m,n)$ est un homomorphisme de
        groupes. On a $a^mb^nc^pa^{m'}b^{n'}c^{p'} = a^{m+m'}b^{n+n'}c^\ast$. Ainsi $|\alpha(g)|_{\alpha(S)}
        \leq |g|_S \leq |g|_S$, c'est-à-dire $|m|+|n| \leq N$. 

        Montrons la seconde inégalité par récurrence sur $N$. Pour $N = 1$, c'est clairement bon. Soit $g' \in
        G$ avec $|g'|_S = N-1$, $g' = a^{m'}b^{n'}c^{p'}$. Alors on a un des trois cas:
        \begin{enumerate}
        \item $g = g' \cdot a^{\pm 1}$;
        \item $g = g' \cdot b^{\pm 1}$;
        \item $g = g' \cdot c^{\pm 1}$.
        \end{enumerate}
        \begin{enumerate}
        \item $g = g'a^{\pm 1} = a^{m'}b^{n'}c^{p'}a^{\pm 1} = a^{m'}b^{n'}a^{\pm 1}c^{p'} = a^{m'}c^{\pm
            n}a^{\pm 1} b^{n'} c^{p'} = \cdots = a^{m'\pm 1} b^{n'}c^{p'\pm n'}$.
            Comme $|p'| \leq (N-1)^2$, on a $|p'\pm n'| \leq (N-1)^2 + |n'| \leq (N-1)^2 + N \leq N^2$. 
            La deuxième inégalité est immédiate par récurrence.
        \item $g'b^{\pm 1} = a^{m'}b^{n' \pm 1} c^{p'}$. Comme $|p'| \leq (N-1)^2$, on a $|p'| \leq N^2$.
        \item $g'c^{\pm 1} = a^{m'}b^{n'}c^{p' \pm 1}$ et donc $|p' \pm 1 | \leq |p'| + 1 \leq (N-1)^2 + 1
          \leq N^2$. 
        \end{enumerate}
      \item Pour l'inégalité $P_1(n) \leq V_S(n)$, montrons que, si $m+n+6[\sqrt{p}] \leq N$ avec $m,n,p\geq
        0$, alors $|a^mb^nc^p|_S \leq N$. En effet, si $k^2 \leq p \leq (k+1)^2$, on a $a^mb^nc^p =
        a^mb^nc^{k^2}c^{p-k^2} = a^mb^n[a^k,b^k]c^{p-k^2}$, donc $|a^mb^nc^p| \leq m + n + 4k + p-k^2 < m + n
        + 4k + 2k + 1 \leq m +  n + 6k \leq N$.
        
        \textbf{Exercice:} $\iiint_{x+y+6\sqrt{z} \leq N,\ x,y,z \geq 0} 1dxdydz = kN^4$ avec $k$ constante.
      \end{enumerate}
    \end{preuve}


    \begin{defi}
      Soient $f, g: \R^+ \to \R^+$. Alors $f \prec g$ s'il existe $a, b, c \in \R$, $c, a > 0$ avec $f(x) \leq
      cg(ax+b)$ pour $x \gg 0$.
      On dit que $f \approx g$ si $f \prec g$ et $g \prec f$.
    \end{defi}

    \begin{exercice}
      Si $f, g$ sont des polynômes de même degré, alors $f \approx g$.
    \end{exercice}

    \begin{prop}[Équivalence de croissance]
      Soient $(G_1, S)$ et $(G_2, T)$ deux groupes quasi-isométriques (par exemples $S, T$ deux parties
      génératrices finies du même groupe $G$). Alors $V_S \approx V_T$.
    \end{prop}

    \begin{preuve}
      Soit $f: G_1 \to G_2$ une quasi-isométrie, c'est-à-dire qu'il existe $\lambda > 0$, $c \geq 0$ telles
      que pour tous $x, y \in G_1$
        \[\frac{1}{\lambda} |x^{-1}y|_S - c \leq |f(x)^{-1}f(y)|_T \leq \lambda |x^{-1}y|_S + c.\]
      On peut supposer $f(1_{G_1}) = 1_{G_2}$ (remplacer $f$ par $(f(1))^{-1}f)$. Donc $\frac{1}{\lambda}|y|_S
      - c \leq |f(y)|_T \leq \lambda |y|_S + c$ pour tout $y \in G_1$. Soit $R \geq 0$ tel que pour tout $h \in
      G_2$, il existe $g \in G_1$ tel que $|f(g)^{-1}h|_T \leq R$ (quasi-surjectivité).
      
      Pour chaque $h$ dans $G_2$, on choisit $g_h \in G_1$ tel que $f(g_h)^{-1}h|_T \leq R$. Pour $h \in
      B_T(n)$, on a $|g_h|_S \leq \lambda (|f(g_h)|_T + c)\ ( \iff \frac{1}{\lambda}|g_h|-c \leq |f(g_h)|) \leq
      \lambda (R+n+c)$ par l'inégalité du triangle. Pour $h$ fixé, le nombre de $h'$ avec $g_h = g_{h'}$ est
      au plus $|B_T(h, 2R) = V_T(2R)$.

      Donc $V_T(n) \leq \left| \{g_h\ |\ h \in B_T(n)\} \right| V_T(2R) \leq V_S(\lambda R + \lambda n +
      \lambda c)V_T(2R)$, qui est exactement la définition de $V_T \prec V_S$.
      
      Par symétrie on a $V_S \prec V_T$, d'où $V_S \approx V_T$.
    \end{preuve}


    \begin{defi}
      \begin{enumerate}
      \item La \emph{suite dérivée} \index{Suite!dérivée} de $G$ est définie par 
        \[G = G^{(0)} \supset G^{(1)} = [G,G] \supset \cdots \supset G^{(k)} = [G^{(k-1)}, G^{(k-1)}] \supset \cdots.\]
      \item La \emph{suite descendante} \index{Suite!descendante} de $G$ est 
          \[G = G_0 \supset G_1 = [G_0, G_0] \supset [G, G_1] \supset \cdots \supset G_k = [G, G_{k-1}]
          \supset \cdots.\]
      \end{enumerate}
      On a donc que $G^{(k)} < G_k$.
    \end{defi}

    \begin{defi}
      \begin{enumerate}
      \item On dit que $G$ est \emph{résoluble} \index{Groupe!résoluble} s'il existe $k \geq 1$ tel que
        $G^{(k)} = \{1\}$.

      \item On dit que $G$ est \emph{nilpotent} \index{Groupe!nilpotent} s'il existe $k \geq 1$ tel que $G_k = \{1\}$.
      \end{enumerate}
    \end{defi}


    On observe ainsi que si $G$ est nilpotent, alors $G$ est résoluble.

    \begin{exs}
      \begin{enumerate}
      \item Tout groupe abélien est nilpotent (donc résoluble).

      \item Pour tout anneau commutatif à unité $A$, le groupe de Heisenberg $Heis(A)$ est nilpotent, car 
          \[[Heis(A), Heis(a)] \subset \left\{
            \begin{pmatrix}
              1 & 0 & z\\ 0 & 1 & 0\\ 0 & 0 & 1
            \end{pmatrix}\ \big|\ z\in A \right\} = Z(Heis(A))
          \]
        le centre de $Heis(A)$ (pour rappel, $Z(G) = \{g \in G\ |\ [g,h] = 1\ \forall h \in G\}$. De plus
          \[Heis(A) \supset [Heis(A), Heis(A)] \supset [Heis(A), \underbrace{[Heis(A), Heis(A)]}_{\subset
            Z(Heis(A))}] = \{1\}\]
        car le centre commute avec tous les éléments.
      \end{enumerate}
    \end{exs}

    \begin{prop}
      Soit $G$ un groupe nilpotent.
      \begin{enumerate}
      \item $G_{j+1} \lhd G_j$
      \item $G_j/G_{j+1}$ est abélien.
      \end{enumerate}
    \end{prop}
    

    \begin{preuve}
      Exercice.
    \end{preuve}







    // AJOUTER LA PARTIE MANQUANTE




    

    En 1983, R. \textsc{Grigorchuck} a constuit le premier exemple d'un groupe $(G, S)$ à \emph{croissance
      intermédiaire}, et a montré que
      \[e^{\sqrt{n}} < V_S(n) < e^{n^{0.991}}.\]

    On a vu (Gromov et Bass) que la croissance polynomiale est équivalente pour un groupe à être virtuellement
    nilpotent. La croissance polynomiale est plus petite que la croissance intermédiaire (Grigorchuck),
    elle-même plus petite que la croissance exponentielle (presque tous les groupes ont une croissance
    exponentielle).




    \begin{defi} \index{Série!de croissance}
      Soit $G$ un groupe finiment avec une partie génératrice finie $S$. La \emph{série de croissance} de $G$
      par rapport à $S$ est
        \[f_{(G,S)}(z) = \sum_{n \geq 0}a_nz^n\]
      où $a_n = \left| \{g \in G\ |\ |g|_S = n\} \right|$ (mots de longueur $n$). On peut aussi parler de $b_n
      = V_S(n) = \left| \{g \in G\ |\ |g|_S \leq n\} \right|$ à la place de $a_n$.
    \end{defi}

    
    \textbf{Question:} existe-t-il des groupes avec une croissance rationelle, i.e. est-ce que $f(z)$ peut
    s'exprimer comme quotient de polynômes $P(z), Q(z) \in \Z[z]$?


    \begin{theo}
      Soit $G$ un groupe, $G = \langle S \rangle$ avec $|S| < \infty$. Si $G$ a un language de formes normales
      qui est régulier, alors la série $f_{(G, S)}(z)$ est rationnelle.
    \end{theo}

    \begin{defi}
      On appelle \emph{formes normales} \index{Forme normale} le fait de prendre un mot sur $S$ pour chaque
      élément dans $G$.
    \end{defi}

    \begin{exs}
      \begin{enumerate}
      \item Considérons $\Z^2$ avec le système de générateurs $\{a^{\pm 1}, b^{\pm 1}\}$. Un langage de
        formes normales est $\{a^nb^m,\ n, m \in \Z\}$.

      \item Considérons le groupe libre $(\F_2, \{a^{\pm 1}, b^{\pm 1}\})$. Un langage de formes normales est
        donné par tous les mots réduits (c'est le langage naturel sur le groupe libre).
      \end{enumerate}
    \end{exs}


    
    

    

    
    



    

    

    


%%% Local Variables:
%%% mode: latex
%%% TeX-master: "../GAC_cours.tex" 
%%% End:

% Chapter 8: Croissance et langages formels
%---------------------------------------------------------%
%______//------             GAC             ------\\______%
%______||------         Chapitre 9          ------||______%
%______\\------   Croissance et langages (formels)   ------//______%
%---------------------------------------------------------%

\chapter{Croissance et langages (formels)}
\label{cha:croiss-et-lang-form}

  \begin{defi}
    Soit $A$ un ensemble fini, et $A^\ast$ l'ensemble de tous les mots formés à partir de $A$. $A$ s'appelle
    un \emph{alphabet}, \index{Alphabet} et $L$ est un \emph{langage} \index{Langage} sur $A$ si $L$ est un
    sous-ensemble de $A^\ast$.
  \end{defi}

  \begin{ex}
    Soit $A = \{0, 1\}$. Alors $L_1 = \{0, 1, 01, 11\}$ et $L_2 = \{0^n1^m\ |\ n, m \geq 1\}$ sont des
    langages sur $A$.
  \end{ex}


  \begin{defi}
    Un \emph{automate fini déterministe (AFD)} \index{Automate!fini déterministe} est un quintuple $(Q, A,
    \delta, q_0, F)$ constitué des éléments suivants.
    \begin{itemize}
    \item $A$ est un alphabet fini.
    \item Un ensemble fini d'états $Q$.
    \item Une fonction de transition $\delta: Q \times A \to Q$.
    \item Un état initial $q_0$.
    \item Un ensemble d'états finaux (ou acceptants) $F \subset Q$.
    \end{itemize}
  \end{defi}

  \begin{ex}
    On peut représenter un AFD par un graphe fini.
    
    \begin{center}
      \begin{tikzpicture}[shorten >=1pt, node distance=2cm, auto]
        % Nodes
        \node[state, initial]   (q0)                         {$q_0$}; 
        \node[state, accepting] (q2) [below right of=q0]     {$q_2$};
        \node[state]            (q1) [above right of=q2]     {$q_1$};
        % Path
        \path[->, >=latex] (q0) edge [bend left]  node {$a$} (q1)
                           (q1) edge [loop right] node {$b$} (q1)
                                edge [bend left]  node {$b$} (q0)
                           (q0) edge [bend right] node {$a$} (q2);
        \draw (6, -1) node{Cet automate n'est pas déterministe};
      \end{tikzpicture}
    \end{center}

    Par convention, une flèche de rien vers un cercle représente l'état initial, un sommet avec deux cercles,
    un sommet avec un seul cercle représente un état,
    représent un état final, $A = \{a, b\}$ et une flèche d'un sommet vers un autre avec une étiquette
    représente une transition et l'ensemble est $\delta: Q \times A \to Q$.

    Un mot $\omega$ est reconnu par un automate s'il existe un chemin étiqueté par $\omega$, partant de l'état
    initial et aboutissant dans un état final.

    L'automate représenté a la figure ci-dessus accepte les mots commençant par $a$, avec au moins un $b$ au
    milieu et terminant par $a$, et le mot $a$. Le langage reconnu par l'automate est l'ensemble des mots
    acceptés par l'automate.
  \end{ex}


  \begin{defi}
    Un langage est \emph{régulier} \index{Langage!régulier} s'il est accepté par un automate.
  \end{defi}

  


  \begin{prop}
    La série de croissance $f_L$ d'un langage $L$ régulier est une fonction rationelle:
      \[f_L(z) = \sum_{n \geq 0}a_nz^n\]
    où $a_n = \left| \{ \omega \in L\ |\ |\omega| = n\} \right|$. 
  \end{prop}

  \begin{preuve}[informelle, preuve par exemple]
    Considérons l'automate $\mathcal{A}$ sur $\{a, b\}$ suivant

    \begin{center}
      \begin{tikzpicture}[shorten >=1pt, node distance=3cm, auto]
        % Nodes
        \node[state, initial, accepting]   (1)                  {$1$}; 
        \node[state, accepting]            (2) [right of=1]     {$2$};
        \node[state]                       (3) [right of=2]     {$3$};
        % Path
        \path[->, >=latex] (1) edge              node {$a$} (2)
                               edge [loop above] node {$b$} (1)
                           (2) edge [bend left]  node {$b$} (1)
                               edge              node {$a$} (3)
                           (3) edge [loop above] node {$a$} (3)
                               edge [loop below] node {$b$} (3);
        
      \end{tikzpicture}
    \end{center}

    Les mots acceptés par $\mathcal{A}$ sont des mots ne contenant pas deux $a$ consécutifs. La \emph{matrice
      de voisinage} \index{Matrice de voisinage} de $\mathcal{A}$ est 
      \[
      M = 
      \begin{pmatrix}
        1 & 1 & 0 \\ 1 & 0 & 1 \\ 0 & 0 & 2
      \end{pmatrix}
      \]
    (on a une transition de $1$ vers $2$, donc $a_{12} = 1$, on a deux transitions de $3$ vers $3$, donc
    $a_{33} = 2$ etc.)

    Pour tous $q, r \in \{1, 2, 3\}$ et tous $n \in \N$, $[M^n]_{q,r}$ est le nombre de chemins de longueur
    $n$ joignant $q$ à $r$.

    \[
    M^3 = 
    \begin{pmatrix}
      3 & 2 & 3 \\ 2 & 1 & 5 \\ 0 & 0 & 8
    \end{pmatrix}
    \]
    par exemple veut dire qu'il y a deux chemins de longueur $3$ joignant $2$ à $1$ ($bbb$ ou $bab$).

    Alors $a_n = \sum_{f \in \{1, 2\}} [M^n]_{1, f}$ pour $n$ fixé. Ainsi la série est
      \[\sum_{n \geq 0}a_nz^n = \sum_{n \geq 0} \sum_{f \in \{1, 2\}}[M^n]_{1, f}z^n = \sum_{n \geq 0}
      [M^n]_{1,1}z^n + \sum_{n \geq 0} [M^n]_{1,2} z^n \]
      \[ = [I + Mz + M^2z^2 + \cdots]_{1,1} + [I + Mz + M^2z^2+\cdots]_{1,2} = [(I-Mz)^{-1}]_{1,1} +
      [(I-Mz)^{-1}]_{1,2}\]
      \[= \frac{1}{\det(I-Mz)} \left( \left[(I-Mz)^{Adj}\right]_{1,1} +
        \left[(I-Mz)^{Adj}\right]_{1,2}\right),\]
    où le déterminant est un polynômes, et les coefficients sont des polynômes. Ainsi la fonction $f_L$ est rationelle.
  \end{preuve}

  
  
  \begin{ex}
    L'automate pour les formes normales de $\F_2$ sur $\{a, A = a^{-1}, b, B = b^{-1}\}$ est représenté ci-dessous.

    \begin{center}
      \begin{tikzpicture}[shorten >=1pt, node distance=4cm, auto]
        % Nodes
        \node[state, initial, accepting]   (0)                  {$1$}; 
        \node[state, accepting]            (1) [right of=0]     {$2$};
        \node[state, accepting]            (2) [above of=0]     {$3$};
        \node[state, accepting]            (3) [left of=0]      {$2$};
        \node[state, accepting]            (4) [below of=0]     {$3$};
        % Path
        \path[->, >=latex] (0) edge              node {$a$} (1)
                               edge              node {$b$} (2)
                               edge              node {$A$} (3)
                               edge              node {$B$} (4)
                           (1) edge [loop right] node {$a$} (1)
                               edge [bend left]  node {$b$} (2)
                               edge [bend right] node {$B$} (4)
                           (2) edge [loop above] node {$b$} (2)
                               edge [bend left]  node {$A$} (3)
                               edge [bend left]  node {$a$} (1)
                           (3) edge [loop left]  node {$A$} (3)
                               edge [bend left]  node {$B$} (4)
                               edge [bend left]  node {$b$} (2)
                           (4) edge [loop below] node {$B$} (4)
                               edge [bend left]  node {$A$} (3)
                               edge [bend right] node {$a$} (1);
        
      \end{tikzpicture}
    \end{center}
    
    Commme on représenter le langage par un automate fini déterministe, le langage est régulier. Ainsi la
    série de croissance de $\F_2$ par rapport à $\{a, A, b, B\}$ est rationelle.

    On a calculé que $a_n = 4 \cdot 3^{n-1}$, et
      \[f_{(G, S)}(z) = 1 + \sum_{n \geq 1} 4\cdot 3^{n-1}z^n = 1 \frac{4}{3} \sum_{n \geq 1}(3z)^n = 1 +
      \frac{4}{3} \frac{1}{1-3z}.\]
  \end{ex}

  

  







%%% Local Variables:
%%% mode: latex
%%% TeX-master: "../GAC_cours.tex" 
%%% End:


\printindex
	
\end{document}



%%% Local Variables:
%%% mode: latex
%%% TeX-master: t 
%%% End: